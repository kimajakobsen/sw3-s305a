\subsection{Model}
We a final class diagram based on the class diagram from chapter \ref{cap:classesevents} so show the design of the system with the MVC utilized. In the current design a few thing has changed:
\begin{itemize}
	\item The role pattern: \\
	The functionality form the actor \admin[] dose not inheritance from \astaff[] and \aclient[] etc. Figure \ref{tab:newactortable} shows the new role system.   
	\item Problem: 
	\begin{itemize}
		\item Deadline: \\
					The deadline is now a attribute instead of a class
		\item tag problem
		\item person problem staff client
	\end{itemize}
	\item 
\end{itemize} 

\begin{figure}[htdp]
\begin{center}
\begin{tabular}{l  ccccc}
\hline 
\multicolumn{2}{r}{\shf{Actor}} \\
\shf{Use case} 			&   \Aclient 	& \Astaff 		& \admin[c]  \\ \hline%
Submit problem 		  	& $\checkmark$ 	&  &  \\ %
My problems 		& $\checkmark$	&   &  \\ %
Worklist 		& 	& $\checkmark$  &  \\ %
Solve problem 			&     			& $\checkmark$	&  \\ %
Administrate			&    			&				& $\checkmark$ \\%
\gstat[c]			&				& 	& $\checkmark$ \\ \hline%
\end{tabular}
\end{center}
\caption{\myCaption{Actor \& use case table}}
\label{tab:newactortable}
\end{figure}

This is a brief description with a list of attributes for each class.


\begin{description}
\item[Problem]\hfill
\begin{description}
\item[Id:]hgh
\item[Title:]hg
\item[Description:]hgh
\item[Added\_date:] hgh
\item[Deadline:]hgh
\item[IsDeadlineApproved:]gh
\item[Reassignable:]h
\item[SolvedAtTime:]hg
\item[PersonId:]hg
\end{description}
\end{description}

\begin{description}
\item[Person]\hfill
\begin{description}
\item[Id:]gh
\item[Name:]gh
\item[Email:]gh
\item[DepartmentId:]gh
\end{description}
\end{description}

\begin{description}
\item[Department]\hfill
\begin{description}
\item[Id:]hg
\item[DepartmentName:]gh
\item[Description:]ghg
\end{description}
\end{description}

\begin{description}
\item[Solution]\hfill
\begin{description}
\item[Id:]gh
\item[Description:]hhg
\end{description}
\end{description}

\begin{description}
\item[Category]\hfill
\begin{description}
\item[Id:]hh
\item[Name:]gh
\item[Description:]
\item[Department\_Id:]df
\end{description}
\end{description}

\begin{description}
\item[Tag]\hfill
\begin{description}
\item[Id:]df
\item[Name:]ff
\item[Description:]df
\item[Priority:]df
\item[Category\_Id:]df
\item[SolvedProblems:]sd
\item[TimeConsumed:]sdd
\item[Hidden:]ee
\end{description}
\end{description}

\begin{description}
\item[Comment]\hfill
\begin{description}
\item[Id:]sf
\item[Time:]fes
\item[Description:]sef
\item[Problem\_Id:]efs
\item[PersonId:]sef
\item[PersonName:]fes
\end{description}
\end{description}

\subsubsection{Classes}
In this section we will give a brief description with a list of attributes for each class, as well as a operating specification of the complex operations.