\subsection{Tools}

The tools classes is a collection of methods that do not display functionallity from the model.

\begin{description}
\item[ProblemDistributer]\hfill
\begin{description}
\item[Purpose:]Distribute problem based on problem tags and staff members workload.
\item[Attributes:]Problem tags, staff id.
\item[Operations:]Distribute problems, reassign problems.
\end{description}
\end{description}

\begin{figure}
\begin{tabular}{p{3.5cm} p{4cm} p{4cm}}
\hline
\textbf{Operation}&Distribute Problems\\
\hline
\textbf{Category}&\underline{ }Active&\underline{x}Update\\
&\underline{x}Passive&\underline{ }Read\\
&&\underline{x}Compute\\
&&\underline{ }Signal\\
\textbf{Purpose}&\multicolumn{2}{p{8cm}}{Distributing problems between staff members in each department, this is done to make sure that a single staff member do not have all the problems.}\\
\textbf{Input data}&\multicolumn{2}{p{8cm}}{Problem id, staff id}\\
\textbf{Conditions}&\multicolumn{2}{p{8cm}}{There must be problems attached to the department calling distribute functions.}\\
\textbf{Effect}&\multicolumn{2}{p{8cm}}{Problems will be distributed equally between staff member until their workload balanced.}\\
\textbf{Algorithm}&\multicolumn{2}{p{8cm}}{The algorithm has to find the staff member with the minimum and the maximum workload, and then distribute their problems between these two staff members, until their workload is balanced. The algorithm has to do this $x$ times where $x$ is the number of people in the department minus one.}\\
\textbf{Datastructures}&\multicolumn{2}{p{8cm}}{Lists}\\
\textbf{Placement}&\multicolumn{2}{p{8cm}}{}\\
\textbf{Involved objects}&\multicolumn{2}{p{8cm}}{Staff id, department id, problem id}\\
\textbf{Triggering events}&\multicolumn{2}{p{8cm}}{Problem added to a department}\\
\hline
\end{tabular}
\morscaption{what a nice figure!}
\end{figure}

\begin{description}
\item[Problem Search]\hfill
\begin{description}
\item[Purpose:]Allows users who is logged on as client or staff to search through problems, the search is based on tags and status.
\item[Attributes:]Tags, status.
\item[Operations:]Search through problems based on tags and / or status.
\end{description}
\end{description}

\begin{figure}
\begin{tabular}{p{3.5cm} p{4cm} p{4cm}}
\hline
\textbf{Operation}&Search Problem\\
\hline
\textbf{Category}&\underline{x}Active&\underline{ }Update\\
&\underline{ }Passive&\underline{x}Read\\
&&\underline{x}Compute\\
&&\underline{ }Signal\\
\textbf{Purpose}&\multicolumn{2}{p{8cm}}{Make is easier for users to look through problems.}\\
\textbf{Input data}&\multicolumn{2}{p{8cm}}{Tag, status.}\\
\textbf{Conditions}&\multicolumn{2}{p{8cm}}{The user must be logged in as either client or staff.}\\
\textbf{Effect}&\multicolumn{2}{p{8cm}}{A page with search results will be shown.}\\
\textbf{Algorithm}&\multicolumn{2}{p{8cm}}{Searches threw the model for all input tags, whereafter it searches again for all input tags minus one, then minus two, etc. until sufficient number of problems has been found.}\\
\textbf{Datastructures}&\multicolumn{2}{p{8cm}}{Lists}\\
\textbf{Placement}&\multicolumn{2}{p{8cm}}{}\\
\textbf{Involved objects}&\multicolumn{2}{p{8cm}}{Tags, status, client or staff}\\
\textbf{Triggering events}&\multicolumn{2}{p{8cm}}{Clients and Staffs}\\
\hline
\end{tabular}
\morscaption{what a nice figure!}
\end{figure}

\begin{description}
\item[SQL]\hfill
\begin{description}
\item[Purpose:]A collection of functions for SQL query and insertion for the aspnet membership database
\item[Attributes:]Username, role name, description.
\item[Operations:]Check if user is in a role, add role to user, check if user is staff, check if user exists, remove role from user, get all roles for a user, get all roles, assign client to user, add a new role, remove user from aspnet membership, reset password, get email for user.
\end{description}
\end{description}

\begin{description}
\item[AverageAllTags]\hfill
\begin{description}
\item[Purpose:]Calculates the average time spent for a problem which has no tags.
\item[Attributes:]\textit{none}
\item[Operations:]averageAll
\end{description}
\end{description}

\begin{description}
\item[MaxPriority]\hfill
\begin{description}
\item[Purpose:] Return the maximum priority which exists in the model.
\item[Attributes:]\textit{none}
\item[Operations:]max
\end{description}
\end{description}