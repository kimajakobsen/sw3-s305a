\chapter{Introduction}
\label{sec:introduction}


%\fixme{synes vi skal snakke om det her, mener selv at den nu udkommenterede start p\aa{} introen er bedre end den nuv\ae rende der st\aa r her.}Organizing and solving problems can be a difficult task. The fact that each service employee need to cooperate with each other to solve the problems, induces a complex organization problem. The goal is to solve all problem as fast as possible. However the problem should be solved in a prioritized order, with the most important problems first. Furthermore should two service employees not try to solve the same problem independently, because this would be redundant work.

%Communication between the solver and the problem giver is a impotent part of solving a problem, the communication can often be   
%Different kind of problems arises when working in a company or an organization. Some of the problem are within your own expertise and you are able to solve them yourself. When a problem occurs and you need help from others, there are a number of things you need to do. 


We live in a world where we usually distribute the problems at hand to the people who are best at solving them. Sometimes this is a easy process, and other times, it can be an immense waste of time trying to find the correct person to hand the problem to. 
Often, we end up solving problems ourselves because it is simply too difficult to find the right person for the job. 
Solving problems ourselves is often inefficient compared to how well an expert in the specific field would do. 
This approach will effectively waste manpower in a company, as it is not the most effective way to solve problems.
Organizing problems can also be difficult for the one who is dedicated to solving them. 
If no systematic approach to organize the problems exists, time might be wasted organizing problems or solving problems that was not important at all.\\
\ \\
The people who are experiencing problems and the people who are solving them presumably have one common goal, which is to solve all problems as fast as possible. This is why large organizations are likely to have dedicated people to solve problems that may arise for the regular employees of the organization.


%\hdesk{} is about solving this particular organizational problem. \hdesk{} purpose is -- in short -- to distribute specific problems to the group of people who are best at solving them, while solving other subproblems such as not overburdening one service employee with all the problems etc.\\
%\hdesk{} does not only serve as a one-way information stream about existing problem, but also delivers a system where the people who solve the problems can report back to the client with solutions or questions regarding the problem. All this while other clients who might have a similar or identical problem, can subscribe and thereby follow the problem from the beginning to the end.\\

%The result of using \hdesk{} this way, is a database full of problems, with -- hopefully -- attached solutions. \hdesk{} uses these informations in a preventive way, which displays possible solutions to a client, and in some cases, these solved trivial problems with attached solutions might help the client instantly, keeping the workload of the people solving problems down, as well as the clients spent time to a minimum.\\
%Productivity goes up, wasted time goes down.

%------------------------------------------------------------------

%Everything needs to be maintained, and as the human race has evolved, we have come to distribute the responsebility of taking care of different kinds of  to differnet people, allowing them to specialize their skill into that particulair field of knowledge. 


%When maintaining an office environment different problems are bound to arise. Computers will break down, printers will need installation and light bulbs will need replacement. Larger organizations will most likely have people hired to do this job exclusively. We develop this system with the purpose of easing the process of solving these problems, as well as distributing knowledge as to how to solve trivial problems that will reoccur, and keeping track of already known problems. Thereby relieving the maintenance staff of these trivial tasks, effectively increasing the productivity of the employees.

