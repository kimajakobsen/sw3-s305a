\section{Development Tools}
In the creation of our web application we use some development tools which will be described in the following section. 

\subsection{IDE}
\label{sub:ide}
The primary development tool has been Microsoft Visual Studio Ultimate \cite{visualStudio}, which include a large variety of inbuilt tools for unit testing, SQL and data modeling. 
None group members had previous experience with development using Visual Studio Ultimate. The alternative were MonoDevelop \cite{mono}, which is a open-source cross platform .NET IDE, but since none had experience with this tool either it were not preferred. 
As most group members work at a daily basis on Windows Based pc's Visual Studio is the favorite choice of IDE. 

%http://www.microsoft.com/visualstudio/en-us/products/2010-editions/ultimate


\subsection{Collaboration}
\label{sub:collaboration}
For collaboration we used Subversion(SVN) and for the code sharing we used AnkhSVN \cite{ankhsvn} together with SVN. 
AnkhSVN is a source control provider for Microsoft Visual Studio. 
Alternatively we could have used Team Foundation Server \cite{teamfoundation}, but this requires installation and configuration of a Team Foundation Server. 
We chose AnkhSVN since we already had a running SVN server. 

%ankhsvn  - http://ankhsvn.open.collab.net/
% Team foundation - http://msdn.microsoft.com/en-us/library/ms181232(VS.80).aspx
\subsection{Database}
\label{sub:database}
As our main data storage we use a Microsoft SQL Server. 
We choose this data storage vendor since it is compatible with ADO.NET Entity Data Model Designer and Visual Studio Ultimate comes with a inbuilt Microsoft SQL Server manager. 
Which allows for editing the SQL server from our workstations and not only from the server itself. 
We considering using postgreSQL, but using postgreSQL with C\# and Visual Studio required a plugin in order to use the ADO.NET Entity Data Model Designer. 
As a database vendor neither choice would not give us any advantages on the data layer, since the required functionality is supported by both systems. 