\section{Distribute Problems}
\label{sec:dispro}
Whenever a new problem is added it needs to be distributed to the correct staff member.
This means the staff in the right apartment with the lowest workload. 
The right department is the department with most categories represented by the submitted problem's tags.
From the analysis we know that a tag belongs to a category which belongs to a department and with this knowledge the algorithm counts the number of times a department is represented in the problem.

There are two special cases which requires some attention. 
The case where two or more departments are equally represented in a problem and the case where a problem has no tags. 
In the first case one is simply chosen by the order in which the department is in the unsorted list of possible departments. 
In the second case the problem is distributed among all staff members and the staff member with the lowest workload will get the problem. 

If a problem is assigned to an inappropriate department we assume that any staff member would have the acquired knowledge to reassign the problem to the appropriate department. 
The problem is always assigned to the staff member with the lowest workload and therefore he has the most time to read the problem and determine the correct department.

The method that does the actual work is the \me{GetStaff} method which has several overloads. 
All overloads takes a \cl{Problem} as parameter. 
The other takes either a department or a list of \cl{IPerson}, which is a \cl{Person} interface made to make it more testable. 
Some overloads use both parameters.