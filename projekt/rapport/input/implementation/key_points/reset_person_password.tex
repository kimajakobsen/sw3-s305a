\section{Reset Person Password}
\label{sec:resetpersonpassword}

The \me{PassMail()} method's -- seen in \ref{lst:passmail} which is part of the \cl{PersonController} controller -- purpose is to reset a specific users password to a new random value, and e-mail it to the specific user. Although not a key feature, this method implements the functionality which sends a user an e-mail, which we consider a key feature as notifying a user is important. The \me{PassMail()} method utilizes the \me{ResetPassword()} method -- seen in \ref{lst:resetpersonpassword} -- which actually generastes the new passwords, and updates the database.

\begin{lstlisting}[style=sourceCode, caption=\myCaption{The ResetPassword(String user) method}, label=lst:resetpersonpassword]
public static String ResetPassword(String user)
{   
    //Connection
    SqlConnection cn = new SqlConnection();
    cn.ConnectionString = connString;
    //Preparing SqlCommand
    SqlCommand password;
    SqlCommand userId;

    //SqlCommand to find UserId
    userId = new SqlCommand("SELECT UserId FROM aspnet_Users WHERE (UserName = '" + user + "')", cn);
    
    //Executing the SqlCommand
    cn.Open();
    String userA = userId.ExecuteScalar().ToString();
    password = new SqlCommand("SELECT Password FROM aspnet_Membership WHERE (UserId = '" + userA + "')", cn);
    String passA = password.ExecuteScalar().ToString();
    //Converting Application into a String
    cn.Close();

    String[] passarray =
    {
        "a","b","c","d","e","f","g","h","j", "i" ,"k",
        "l","m","n","o","p","q","r","s","t","u",
        "w","x","y","z","A","B","C","D","E","F",
        "G","H","I","J","K","L","M","N","O","P","Q",
        "R","S","T","U","V","W","X","Y","Z",
        "0","1","2","3","4","5","6","7","8","9"
    };

    String setPass = "";
    //Preparing a two randoms, one for password length and one for picking a char from passarray.
    Random RandomNumber = new Random();
    Random RandomPass = new Random();
    int x = RandomNumber.Next(10,25);
    //Putting the passowrd together
    for (int i = 0; i < x; i++)
    {
        int y = RandomPass.Next(passarray.Length);
        setPass += passarray[y].ToString();
    }
    //Finding the user
    MembershipUser u = Membership.GetUser(user);
    //Generating a new password for the user
    String np = u.ResetPassword();
    //Changing the password for the user by using the the resetted password as old password
    u.ChangePassword(np, setPass);
    return setPass;
}
\end{lstlisting}

\begin{lstlisting}[style=sourceCode, caption=\myCaption{The PassMail(int id) method}, label=lst:passmail]
[Authorize(Roles = HoplaHelpdesk.Models.Constants.AdminRoleName)]
public ActionResult PassMail(int id)
{
    //Getting name from person user
    var person = db.PersonSet.FirstOrDefault(x => x.Id == id);
    String user = person.Name.ToString();
    try
    {
        //Used for testing
        string msg = HttpUtility.HtmlEncode("Person.PassMail, User = " + user);
            
        //Preparing mail
        MailMessage mail = new MailMessage();
        //Setting up a smtp client
        SmtpClient SmtpServer = new SmtpClient("smtp.gmail.com");
        //Adding a from mailaddress
        mail.From = new MailAddress("helps305a@gmail.com");
        //Adding user to maillist, by using getmail function.
        mail.To.Add(SQLf.GetEmail(user));

        //Adding a subject and a message for the mail
        mail.Subject = "Hopla Helpdesk: Your password has been changed!";
        mail.Body = "Username: " + user + "\nPassword: " + SQLf.ResetPassword(user) + "\n\nDo not reply to this email.\n--\nKind Regards\nHopla Helpdesk - Staff Team";

        //Config the Smtpserver details
        SmtpServer.Port = 587;
        SmtpServer.Credentials = new System.Net.NetworkCredential("helps305a", "trekant01");
        SmtpServer.EnableSsl = true;

        //Sending mail
        SmtpServer.Send(mail);

        //Adding view if mail was success
        ViewData["Success"] = "The users password was reset to a random value and emailed to the user.";
        ViewData["View"] = "Index";
        return View("Success");
    }
    catch
    {
        //An error view will be added if an error occurred
        ViewData["Error"] = "The password cannot be resetted, " + user + " don't have a mail";
        ViewData["View"] = "Index";
        return View("Error");
    }
}
\end{lstlisting}