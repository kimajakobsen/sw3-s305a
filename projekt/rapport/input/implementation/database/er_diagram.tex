\newcommand{\erdiagram}[1][]{E-R diagram}
\section{E-R Diagram}
In order to get an overview of how to structure the database \erdiagram[]s gives a neat foundation. 
The \erdiagram[] is based upon the class diagram \fixme{Inds\ae{}t en reference til det f\o{}rste klasse diagram her!!}. 
Every class is turned into a entity and every relation is turned into a relationship. 
The three classes who inherits from each other, \client[], \staff[], and \admin[] are combined into one entity named roles. This is due to that a person can only have one role, and therefore this representation is more efficient. This also gives a nice modifiability because adding a new role is simply just adding a new tuple to the entity. 