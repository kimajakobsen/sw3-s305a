\section{Mapping ER-diagram to relational scheme}
\label{sec:map_er_rel}

Our relations will be as follows:\\

\noindent\textit{Problem (\underline{problemID}, name, description, date, eta, deadline)} \\

\noindent\textit{Person (\underline{personID}, name, e-mail, username, password, dept\_name)}  -- Note here that we have included \verb+dept_name+ in \verb+Person+ as we have a many-to-one relation to the \verb+Department+ scheme, and therefore the redundant table called \verb+Employment+ can be omitted.\\

\noindent\textit{Solution (\underline{solutionID}, description, date)} \\

\noindent\textit{Department (\underline{deptID},dept\_name, description)} \\

\noindent\textit{Tag (\underline{tagID}, name, description, priority, categoryID)} -- Note here that we have included \verb+categoryID+ as we have a many-to-one relation to the \verb+Category+ scheme, and therefore the redundant table called \verb+tag_cat+ can be omitted.\\

\noindent\textit{Category (\underline{name}, \underline{description}, dept\_name)} -- Note here that we have included \verb+dept_name+ in \verb+Category+ as we have a many-to-one relation to the \verb+Department+ scheme, and therefore the redundant table called \verb+dept_cat+ can be omitted.\\

\noindent\textit{Role (\underline{roleID}, name)} \\

\noindent\textit{Comment (\underline{time}, \underline{description}, problemID, personID)} -- Note that we have included \verb+problemID+ and \verb+personID+ as we have a many-to-one relationship with the \verb+Problem+ scheme, and therefore the redundant tables called \verb+com_prob+ and \verb+made_by+ can be omitted.\\

\noindent\textit{Assignment (personID, problemID)} \\

\noindent\textit{Subscription (personID, problemID)} \\

\noindent\textit{prob\_tag (problemID, tagID)} \\

\noindent\textit{prob\_sol (problemID, solutionID, time)} \\

\noindent\textit{per\_rol (personID, roleID)} \\