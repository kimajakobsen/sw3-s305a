\chapter{Conclusion}
\label{chap:conclusion}
We initiated this project with a partly comprehensive project proposal which meant that we could start the analyzation-phase immediately. This went hand-in-hand with the \ooad{} method which the study regulation required that we used. The \ooad{} method resulted in a comprehensive analysis and design phase, which we documented immediately afterwards. We afterwards chose to useuse the MVC design pattern, which -- due to the programming language of choice, C# the ASP.NET MVC 2 framework, which took a relatively big amount of time to learn. During this learning phase, we realized that we had to redo and change alot of the decisions we made during the design phase, as well as reinterpret some parts from the analysis, in order to use the ASP.NET MVC 2 framework for our implementation. Due to the limited amount of time we had left, we were forced to make these decisions, which not always led to a better and more intuitive application. This led to a rather large improvements chapter (chapter \ref{chap:future_work}). We did not document our developed application until we decided that we were done developing, whereafter the remaining time were spent documenting our application, as well as decisions which were made during the development part.


%The goal of this project is to use the knowledge gained from the two courses SAD and OOP. The approach to achieve this is to make a application based upon the materials learned from the two courses. This resulted in a throughly analysis and design instead of spending time of learning how the ASP.NET MVC2 framework works. Which meant that a lot of the time we initially spent on the design were completely wasted as we realized that it could not be implemented properly using the framework. Therefore we started very late on the implementation and did not use the required time to throughly make a full system test and correct all bugs. This could be avoided by achieving knowledge of the framework prior to the design phase. 




%In general we used a lot of new technologies which we had a very sparse knowledge of and this resulted in parts of the system were we afterwards can see that this could have been done more comprehensible and efficient. The whole learning process of this project has been trial and error. We tried applying new technologies and sometimes it went well or if not we learned something from it. 


