\chapter{Conclusion}
\label{chap:conclusion}
This chapter concludes whether or not our application live up to our application definition, which is seen in section \ref{sec:systemdefinition}.

Our front end part of our application -- which the users of the application uses -- is web based, and can run in a browser, thereby enabling the users of our application without any installation, assuming they have a browser.

Our application contains tags which have priories.
These priorities are used to calculate the importance of a problem.
The priority of a given problem is found by calculating the average priority of all tags attached to the problem.

The problems each has an estimated completion time.
An estimation called estimated time consumption is used to calculate the estimated time of completion.
The estimated time consumption is based on the time which the tags that are attached to the given problem takes to solve on average.
If a problem is not on the top worklist of the the \astaff[] member who is assigned to solve it, the estimated time consumption from every problem above the given one is summed up and added to the estimated time consumption of the given problem, this yields the estimated time of completion for the given problem.
The way the estimated time of completion is calculated is described in further details in section \ref{sec:expected_time_of_completion} and the estimated time consumption is described in section \ref{sec:estimated_time_consumption}.











%%%%%%%%%%%%%%%%%%%%%%%%%%%%%%%%%%%%%%%%%%%%%%%%%%%%%%%%%%%%%%%%%%%%%%%%%%%%%%%%%%%%%%%%%%%%%%%%%%%%%%%
\begin{comment}
We initiated this project with a comprehensive project proposal -- seen in appendix \ref{app:projectProposal} -- which meant that we could start the analysis-phase immediately.
This went hand-in-hand with the \ooad{} method which we have been taught in the SAD course.
The \ooad{} method resulted in a comprehensive analysis and design phase, which we documented immediately afterwards.
We then chose to use the MVC design pattern, which -- due to the programming language of choice, C\# the ASP.NET MVC 2 framework, which took a relatively large amount of time to learn.
During this learning phase, we realized that we had to redo and change a lot of the decisions we made during the design phase, as well as reinterpret some parts from the analysis, in order to use the ASP.NET MVC 2 framework for our implementation.
Due to the limited amount of time we had left, we were forced to make these decisions, which not always led to a better and more intuitive application.
This led to a rather large improvements chapter; \ref{chap:future_work}.
We did not document our developed application until we decided that we were done developing, where after the remaining time were spent documenting our application, as well as decisions which were made during the development part.
We reflected upon our process and choices, and suggested ideas for if we were to start over.

%System def
%criterion 



%The goal of this project is to use the knowledge gained from the two courses SAD and OOP. The approach to achieve this is to make a application based upon the materials learned from the two courses. 
\end{comment}
%%%%%%%%%%%%%%%%%%%%%%%%%%%%%%%%%%%%%%%%%%%%%%%%%%%%%%%%%%%%%%%%%%%%%%%%%%%%%%%%%%%%%%%%%%%%%%%%%%%%%%%