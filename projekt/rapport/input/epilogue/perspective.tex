\chapter{Perspective}
\label{chap:perspective}
%\myTop{top}
The goal of the project was to create a general helpdesk which should make it easy and fast for persons get their problems solved, and in the same time lighten the work burden of the service employees by organizing the problems and spare them from solving trivial reappearing problems. Even though the problem seems real then we did not examen if there existed such a need in the real world. We did not base our system definition on interviews of service personal but based it our experience in the area. 

The system was designed to be a general system which should be usable in all kinds of organizations and companies. Because the system was never completed and never tested in a real life company, we have no way of knowing if the structure would cover every companies and organizations needs. Furthermore did we not prioritize usability testing of \hdesk[] and, therefore, a lot of real life functionality could be missing without our knowledge of it.   

Our \hdesk[] is only able to help persons electronically. Therefore should phone-, person to person-, and letter-contact be avoided because the \hdesk will the not be efficient.    

In short, the \hdesk[] should make the work process easy and not complicate the process, to insure this is it very important that the systems configuration is setup correctly. It is of the utter most important that the users adapt as much as possible to the system and then the system should be flexible enough to smooth the work process.   

%Since the main goal of the project is to achieve knowledge we did not spent any time talking to possible costumers or the staff members of the university who is supposed to benefit from the system. From the beginning we have not assumed they will be using the system and therefore we made the requirements and analysis based upon our idea and previously experience from working at similar places. 
%If did was not a learning process we would have used possible costumer to properly analyze the problem domain. 

%\myTail{tail}