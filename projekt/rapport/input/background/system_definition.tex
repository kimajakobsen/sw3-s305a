\section{System Definition}
\label{sec:systemdefinition}


A webbased system used to ease the collaboration between client, who has a problem, and technical staff, related to solving the problem, where the main empasis is on getting the problems to the persons who are best at solving them, as efficient as possible.
%The system will automatic reassign problems if a staff member has become overloaded.
Before a problem is actually submitted to the system, the client is asked to \textit{categorise} the problem by selecting tags from categories wich define and describe the problem as specific as possible. Based on this categorization, the system will search for similair problems which solution might help the client, and theirfore remove the need for submitting an identical or almost identical problem to the system.

If the client finds a problem wich are equal or identical to his/hers own problem, but without a solution, the client is presented with the option to \textit{subscribe} to the problem, and receive notifications whenever new comments are made or a solution gets attached to the problem.

If the client should not find a solution to his/hers problem, he/she would be allowed to submit the problem, which would be assigned to an appropriate staff member, based on the clients selection of tags and categories, and the current workload of the staff members.

If a client can not determine the correct categorisation for a specific problem, it will simply be submitted without the categorization, and the system will from there do one of the following three things, which a system administrator specifies:

\begin{itemize}
	\item distribute between all staff members, based on their workload.
	\item distribute between all staff-personel associated with a specific department, based on their workload.
	\item distribute to a single staff member, regardless of his workload.
\end{itemize}

Every time a problem is assigned to any staff member, he or she will get a notification about it.

The system will generate priority sorted to-do lists for the staff. The priority will be based on a suggested priority by the client, and a final priority evaluation by the staff member assigned to the problem.
%The system will generate priority sorted to-do lists for the staff. The priority will be based on the category and the time since it were committed.

The system will make estimations based on time for solving problems of the same or somewhat equal categorization, the current workload of the assigned staff member, and the priority of the specific problem versus the other problems of which the specific staff member has to solve as well.
%% The estimation will be used to estimate a deadline. The staff member assigned to the problem will be notified when the deadline is near.\\

When problems are commited to the system and assigned to a staff member, the staff and client can add notes to the problem as a way of communication. Every time a change is made to a problem or a comment has been added, all associated clients and the assigned staff member gets notified.


DON'T READ PAST THIS LINE\\




===========
A webbased system used to ease the collaboration between client, who has a problem, and technical staff, related to solving the problem, where the main empasis is on getting the problems to the persons who are best at solving them, as efficient as possible.
 
Clients can submit problems which will be assigned to a staff member, based on the clients selection of tags and categories, the current workload of the staff members and statistics for time taken for solving similar problems for the given staff member. 
The system will automatic reassign problems if a staff member has become overloaded.
The system will generate priority sorted to-do lists for the staff. The priority will be based on the category and the time since it were committed.
The system will make estimations based on time for solving problems of the same category, the current workload of the assigned staff member, and the priority. The estimation will be used to estimate a deadline. The staff member assigned to the problem will be notified when the deadline is near.\\

This estimation will be used as deadline and near deadline the staff member will be notified.

If a client can not determine the correct category for a specific problem, he will assign it to the department he feels will be best equipped for the task. The system will then assign the problem to a department supervisor who determines the priority of the problem and assigns it to the best suited staff member. 

When client submits a problem the system will check the problem against old problems; if a similar problem previously has been solved the system will show this solution to the client. If that particular solution is insufficient the client will be granted the opportunity to submit it. The staff members should select if they find the solution important to be saved.

When problems are commited to the system and assigned to a staff member, the staff and client can add notes to the problem as a way of communication. Every time a change is made to a problem or a notes has been added the client is notified\\
===========





FACTOR criterion\\
F (functionallity): Rankings, keep track, knowledge saved, Statistic time estimation, generate todo-lists.

A (Application domain): People at the university

C (Conditions): Only people associated with university, a webbased system, only people associated with the university, to-do-lists must be agile

T (Technology): Web based, 

O (Objects): Problems, solotions, staff, supervisor, client

R (Responsibility): An administrative communication medium to ease communication between staff and client.\\










A webbased system for the university used to ease the collaboration between client, wich has a problem, and technical staff or other poeple related to solving the problem. The "magic" happens as the system efficiently, cleverly and reasonably administrates wich problems to be solved when, by whom and/or where. The highest purpose of the system is to get problems solved as efficiently as possible.\\



A problem can have multiple states, to indicate how far the problem is from being solved.
The system must keep track of problems in order for the user to be able to see how far along the problem from being solved. The user should be able to rank the submitted problem, so that more important problems can be solved faster. The administrator should be able to change this rank at any point. 
The system should be able to assign problems to the members of the technical staff based on the current workload and the problem rankings. These lists must be agile, meaning that you must be able to easily change it in order to enhance efficiency.
The system will save solved problems in order to use the knowledge achieved from solving the problem. This knowledge could be usable for the technical staff in order for him to see how other staff members has solved similar problem. The system should support ways of communication/collaboration between the user and the administrator, in case further information is needed. The system should be able to estimate how long before a problem is solved.
