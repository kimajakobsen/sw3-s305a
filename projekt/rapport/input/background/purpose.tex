\section{Purpose}
When maintaining an office environment different problems are bound to arise. Computers will break down, printers will need installation and light bulbs will need replacement. Larger organizations will most likely have people hired to do this job exclusively. We develop this system with the purpose of easing the process of solving problems, as well as distributing knowledge as to how to solve trivial problems that will reoccur, and keeping track of already known problems. Thereby relieving the maintenance staff of these trivial tasks, effectively increasing the productivity of the employees.
\section{Introduction}
\label{sec:introduction}


\fixme{L\ae s v\ae nligst dette igennem, og slet denne fixme, hvis du synes dette stykke tekst er fint.}
We live in a world where we distribute the problems at hand to the people who are best at solving them. Sometimes this is a easy process, and other times, it can be an immense waste of time trying to find or contact someone who we can ask. Often, we end up solving problems ourselves, which someone else could have solved in, let us say one fifth of the time we used, at a lower cost. Not only did we waste our time, but we could have spent the time working with other things which we are particularly good at, keeping our work efficiency at a maximum.\\

\hdesk{} is about solving this particular problem. \hdesk{} purpose is -- in short -- to distribute specific problems to the group of people who are best at solving them, while solving other subproblems such as not overburdening one individual with all the problems etc.\\
\hdesk{} does not only serve as a one-way information stream about existing problem, but also delivers a system where the people who solve the problems can report back to the client with solutions or questions regarding the problem. All this while other clients who might have a similar or identical problem, can subscribe and thereby follow the problem from the beginning to the end.\\

The result of using \hdesk{} this way, is a database full of problems, with -- hopefully -- attached solutions. \hdesk{} uses these informations in a preventive way, which displays possible solutions to a client, and in some cases, these solved trivial problems with attached solutions might help the client instantly, keeping the workload of the people solving problems down, as well as the clients spent time to a minimum.\\
Productivity goes up, wasted time goes down.

Comprehensive knowledge of the OOA\&D method is assumed. \fixme{skriv dette ind i introen}


%Everything needs to be maintained, and as the human race has evolved, we have come to distribute the responsebility of taking care of different kinds of  to differnet people, allowing them to specialize their skill into that particulair field of knowledge. 


%When maintaining an office environment different problems are bound to arise. Computers will break down, printers will need installation and light bulbs will need replacement. Larger organizations will most likely have people hired to do this job exclusively. We develop this system with the purpose of easing the process of solving these problems, as well as distributing knowledge as to how to solve trivial problems that will reoccur, and keeping track of already known problems. Thereby relieving the maintenance staff of these trivial tasks, effectively increasing the productivity of the employees.

