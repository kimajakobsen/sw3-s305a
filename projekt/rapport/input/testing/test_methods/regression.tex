\section{Regression Testing}
Once the unit tests have been run and passed, the program is not necessarily done.
If a feature is altered (optimized, refactored etc.) later, tests should be run again to ensure that the alteration did not break any functionality. This type of testing is known as regression testing.

Usually only test cases regarding the component being changed are run, since running every test case will usually take a large amount of time.
However, we do not have that many test cases, so we would run every test case whenever a major component was changed.
By doing so we ensure that our program will never regress.

\fixme{Spørg Nadeem om dette statement er valid}
The regression tests we have ran are used on white box unit tests, but upon running them as regression tests, they are black box tests, because we do not make a new analysis of the changed component.