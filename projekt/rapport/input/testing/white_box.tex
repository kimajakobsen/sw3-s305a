\chapter{White Box Testing}
\label{chap:whiteBox}
\myTop{This chapter will describe what white box testing is and how we have used it to test our system.
It is important for us to test our system, to make sure that it works accordingly to what we expect.}
\section{Definition}
White box testing requires knowledge of the code being tested.
It involves finding every code path the specific component can take, e.g. every if statement should be tested where the condition is both true and false.
It is also need to test the edge conditions for the the loops within the code of the component being tested

This is done by making a flow chart of the component being tested to see which code paths there are.
The flow chart can then be used as a graph and then every path from the start of the component to the end can be found using depth first search which will run until every vertex is in a path leading from the start to the end vertex.
Every path is recorded so that a test case covering every path can be created.
\cite{whiteBox}

\section{Our Use of White Box Testing}
We have used white box testing to test our tool component and the parts of the model component which we have wrote our selves -- the ADO.NET was used to generate much of the needed model.

\myTail{}