\chapter{Unit Testing}
\label{chap:testing}
\myTop{This chapter outlines the testing we did during and after the development of our system, as well as an example. We have utilized testing as correctness of the key methods are crucial to our system.}

Throughout the development, we've used Visual Studios built in Team Test, which is an integrated unit-testing framework. \cite{teamtest} \\
The idea behind unit testing is to check an individual method by executing it with approproate input, and afterwards check that its output corresponds to what it should.

All major functions have been tested with Team Test, an example is shown in code snipper \ref{lst:balanceWorkloadTest19}. The example unit test creates four problems, all attached to four individual tags, all with the same time \verb+TimeConsumed+ setting. Two persons are then created, where one of them gets three out of the four problems, and the last person gets the remaining problem. The \verb+BlanceWorkload+ method is then used, to balance the workload of the two persons. Afterwards, the test checks that the problems have been evenly distributed between the persons.


\begin{lstlisting}[style=sourceCode, caption=\myCaption{An example unit test which tests a specific instance of the balanceWorkload method.}, label=lst:balanceWorkloadTest19]
[TestMethod()]
public void BalanceWorkloadTest2()
{
    var tag1 =  new Tag(){ TimeConsumed = 20, SolvedProblems = 1 , Priority = 1  };  //(TimeConsumed / SolvedProblems) = 20
    var tag2 =  new Tag(){ TimeConsumed = 20, SolvedProblems = 2 , Priority = 2  };  //(TimeConsumed / SolvedProblems) = 10
    var tag3 =  new Tag(){ TimeConsumed = 20, SolvedProblems = 2 , Priority = 3  };  //(TimeConsumed / SolvedProblems) = 10
    var tag4 =  new Tag(){ TimeConsumed = 20, SolvedProblems = 2 , Priority = 4  };  //(TimeConsumed / SolvedProblems) = 10

    var prob1 = new Problem() { Tags = new EntityCollection<Tag> { tag1 }, Reassignable = true };
    var prob2 = new Problem() { Tags = new EntityCollection<Tag> { tag2 }, Reassignable = true };
    var prob3 = new Problem() { Tags = new EntityCollection<Tag> { tag3 }, Reassignable = true };
    var prob4 = new Problem() { Tags = new EntityCollection<Tag> { tag4 }, Reassignable = true };
   
    var mike = new Person() { Name="mike", Worklist = new EntityCollection<Problem>() { prob1, prob2, prob3 } }; // Workload = 40
    var john = new Person() { Name= "John", Worklist = new EntityCollection<Problem>() { prob4 } };               // = 10

    Department target = new Department()
    {
        Persons = new EntityCollection<Person>()
        {
            mike, john
        }
    };
    target.BalanceWorkload();
    Assert.IsTrue(john.Worklist.Contains(prob1));
    Assert.IsTrue(mike.Worklist.Contains(prob2));
    Assert.IsTrue(john.Worklist.Contains(prob4));
    Assert.IsTrue(mike.Worklist.Contains(prob3));
}
\end{lstlisting}

\myTail{This chapter outlines the testing we did during and after the development of our system, as well as an example.}