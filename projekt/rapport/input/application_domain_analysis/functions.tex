\section{Function}
\label{sec:function}

The purpose of this section is to ``determine the system's information processing capabilities'' \cite[p.~137]{roedeaalborg}. Ultimately resulting in ``a complete list of functions with specification of complex functions.'' \cite[p.~137]{roedeaalborg} 
All functions with a medium or complex complexity are explained below.



\begin{figure}[hpt] %tdp t = top , p = own page, d = unknown, b = bottom, h = here
\begin{center}
\begin{tabular}{|l|l|l|}
\hline
\textbf{Name}								&\textbf{Complexity} & \textbf{Operations}   \\ \hline%
%Get all categories 					&   Simple & Read   \\ \hline%
Get problem tags						&   Simple & Read   \\ \hline%
Search for problems 				&  Complex & Read/Calculate   \\ \hline%
Create problem 							&   simple & Update   \\ \hline%
%Get problem matching tags 	&   simple & Read/Calculate   \\ \hline%
%Get problem tags 						&   Simple & Read   \\ \hline%
%Get / set problem Deadline 	&   Simple & Read/Update  \\ \hline%
%Get / set problem Title 		&   Simple & Read/Update   \\ \hline%
%Get / set problem Description & Simple & Read/Update   \\ \hline%
%Get / set problem status 		&   Simple & Read/Update   \\ \hline%
%Get / set problem added date& 	Simple & Read/Update   \\ \hline%
%Get / add problem solutions &   Simple & Read/Update/Signal   \\ \hline%
%Get / add problem comments 	&   Simple & Read/Update/Signal   \\ \hline%	
%Get / set problem priority 	&   Simple & Read/Update   \\ \hline%
Reassign staff to problem 	&   Simple & Read/Update/Signal   \\ \hline%
Subscribe / unsubscribe  to problem 	&   Simple & Update   \\ \hline%
%Get problem assigned staff 	&   Simple & Read   \\ \hline%
Attach / detach solution to problem	&   Simple & Read/Update   \\ \hline%
Manage Tag Times &   Medium & Update/Calculate \\ \hline
Get Problem Est. Time Consumption 		&   Medium & Read/Calculate   \\ \hline%
Manage Tag Times &   Medium & Update/Calculate \\ \hline
Get Problem Est. Time Consumption 					&   Medium & Read/Calculate   \\ \hline%
Get Problem Expected Time of Completion 					&   Medium & Read/Calculate   \\ \hline%
Get Tag average time consumption					&   Simple & Read/Calculate   \\ \hline%
%Get worklist 								&   Simple & Read   \\ \hline%
Get sorted worklist 								&   Medium & Read/Calculate   \\ \hline%
Approve deadline 						&   Simple & Update   \\ \hline%
%Get list of departments 		&   Simple & Read   \\ \hline%
Create department 					&   Simple & Update   \\ \hline%
Create category 						&   Simple & Update   \\ \hline% 
Create tag									&   Simple & Update   \\ \hline%
Hide / show category				&   Simple & Update   \\ \hline%
Hide / show tag							&   Simple & Update   \\ \hline%
%Get departments staff 			&   Simple & Read   \\ \hline%
%Get departments category 		&   Simple & Read   \\ \hline%
%Get / set person email 			&   Simple & Read/Update   \\ \hline%
%Get / set person department &   Simple & Read/Update   \\ \hline%
%Get / set person role 			&   Simple & Read/Update   \\ \hline%
Delete person 							&   Simple & Update   \\ \hline%
%Get person name 						&   Simple & Read   \\ \hline%
%Get persons workload 				&   Simple & Read   \\ \hline%
Reset person password 			&   Medium & Update/Signal   \\ \hline%
Balance workload 						&   Complex & Calculate   \\ \hline%
Get statistics							&   Medium & Read/Calculate   \\ \hline%
Distribute Problem & Medium & Read/Calculate/Update \\ \hline%
\end{tabular}
\end{center}
\morscaption{Function list}
\label{tab:functionlist}
\end{figure}

\paragraph{Search for problems }
This function searches the model for problems with specific tags, then presents them in an ordered list assorted after problems with most similar tags. 
It is both \textit{read} and \textit{calculate} because it reads in the model and compute which problems are most similar. 
This function is \textit{complex}.

\paragraph{Get Estimated Time Consumption of Problem }
This function calculates the amount of time a specific problems acquires to be solved. This is based upon the related tags. 
The tags has a property describing the average time consumption.  

\paragraph{Get Expected Time of Completion of Problem}
This function takes all the problems of the assigned staff member which are expected to be solved prior this problem and adding up the amount each is estimated time each problem will consume.

\paragraph{Get Average Time Consumption of a Tag }
This function calculates the average time consumption of a tag. 
It has to divide the number of problem solved and the total amount of time all problems with this tag required to be solved. 


\paragraph{Get Sorted Worklist}
This function returns the work list of a staff member in sorted order after priority. 

\paragraph{Manage Tag Time of Problem }
This function takes the time a staff has used to solve a problem and adds to the related tags time consumption and increments the solved problems property. 

\paragraph{Reset person password}
If a person forgets his password, then he can get a new password send to his mail. The password is randomly generated. This function updates the model to match the computed password. This function is \textit{medium}.

\paragraph{Balance workload}
This function compares the workload of staff and distributes their problems equally among other \astaff members in the same department. This function calculate the workload of \astaff s and then moves problems in the most optimal way. This function is Complex.   

\paragraph{Get statistics}
The statistics function calculate how long each \astaff[] use to solve a problem in average. The function can also find the total average for departments. By reading values form the model and calculating the best way to balance the workload, problems will be solved faster. This function is medium.










%%%%%%%%%%%%%%%%%%%%%%%%%%%%%%%%%%
\begin{comment}
\paragraph{Add Comment} This function simply adds a comment to an existing problem, making the comment visible for other \client{} and \astaff[] members. This function is marked as an \textit{signal}-function because a notification is sent to the assigned \astaff[] member.

\paragraph{Reassign Problem} The reassign problem function reassigns a specific problem from one assigned \astaff[] member to another. It is marked as a \textit{signal}-function as the new assigned \astaff[] member received a notification of the newly reassigned problem. 

\paragraph{Change Problem Status} This function changes the status of a given problem. It is marked as a \textit{singal}-function since the associated \client s receives a notification. 

\paragraph{Delete Problem} To delete a problem, this function is used.

\paragraph{Search Problems} 
This function searches the model for problems with specific tags. It is both \textit{read} and \textit{compute} because it will compute which problem(s) compares best to the tags which are being searched for. This function is \textit{complex} because of the ordering of the found problems.
%searches the model for problems matching a set of tags. Marked as ``complex'' complexity due to somewhat advanced algorithms.\fixme{Er det virkelig derfor den er medium?} Marked as a \textit{read}-function as it simply reads the model.

\paragraph{Create Solution} This function simply creates solution to a problem and adds it to the model. 

\paragraph{Attach Solution} This function attaches an existing solution to a problem, and notifies the associated \client s about it. This function is a \textit{signal}-function as it notifies the associated clients. 

\paragraph{Distribute Problems} The distribute problem function is ran everytime a new problem is created and by a given interval to ensure that no staff is overcrowded with tasks. 
It distributes newly created problems by assigning them to an appropriate \astaff[] member, based on current workload. % and specifications on what the different \astaff[] members are good at.
Marked as a \textit{read}-function as it reads from the model. Marked as a \textit{signal}-function as it notifies the associated clients, and the newly assigned \astaff[] member. 

%\paragraph{Compare Problems} compares a specific problem to the problems in the model and returns similair problems, prioritizing the ones with an attached solution, hence complex in complexity. Marked as a \textit{read}-function as it reads from the model, and marked as a \textit{compute}-function as a computation is involved involving information provided by the \client. \\

\paragraph{Create Problem} This function creates a new problem and adds it to the model. Every time this function is ran, it computes which department is best suited for solving the problem, and thereafter assignes the problem to the \astaff[] member which has least problems to solve. 

%\paragraph{Attach user to problem} simply attaches a new user to an existing problem. The need for this function arises when a \client has a problem which is in need of a solution, but the problem has already been entered in the system by another \client. \\

\paragraph{Get Staff Todo List} This function simply fetches the \astaff[] members todo list. Hence marked as a \textit{read}-function. 

\paragraph{Create Tag} creates a tag, which is an element used for categorizing problems, which purpose is to make searching easier, also attaches the tag to a category.

\paragraph{Create Category} To create a category, this function is used. It also attaches that category to a department. 

\paragraph{Delete Tag} If a tag is not attached to any problem it can be deleted, if not it can only be hided. 

\paragraph{Delete Category} If any tag is attached to the attempted deleted category it can not be deleted and can only be hidden. 

\paragraph{Change Visibility of Category} The Change Visibility of Category function changes the visibility of a category. Changes happen and a need of a category may vanish over time, but might be needed again in the future, thus making this function necessary. 

\paragraph{Change Visibility of Tag} This function hides tag if the \sadmin{} no longer finds it useful. In cases a delete can not be done, because the \textit{tag} is related to some problems. 

%\paragraph{Attach tag to category} simply attaches a specific tag to a specific category. 

%\paragraph{Detach tag from category} simply detaches a specific tag from a specific category. 

%\paragraph{Reattach tag} reattaches a specific tag from it's current category to another specified category. Marked as an \textit{update}-function as it updates the model.

\paragraph{Create Department} All \astaff[] members are associated with a department in order to ease the process of distributing newly created problems to appropriate \astaff[] members. 

\paragraph{Delete Department} As the name dictates, this function simply deletes a department. 

\paragraph{Add Person} A person can either be a \aclient[], \astaff[], or an \admin[]. See figure \ref{tab:actoreventtable} in section \ref{sec:actors}. This function creates a person with characteristics to define him/her in the system.

\paragraph{Rename Person} This is a trivial function, which simply changes the name of a person.

\paragraph{Remove Person} This function removes a person from the system.

%\paragraph{Assign staff to department} assigns a specific \astaff[] member to a specific department. 

\paragraph{Get Statistics} This function returns statistics, e.g. average time for specific kinds of problems to be solved, hence ``complex''-complexity. It is marked as a \textit{compute}-function since it involves a fair amount of computation on the data in the model.

\paragraph{Subscribe Client} To tie a client and a problem together through a subscription, this function is used. It is marked as a \textit{signal}-function since the assigned \astaff[] member receives a notification. 

\paragraph{Set Priority} This function sets the priority of a tag. 

\paragraph{Get Priority} This functions computes the priority of a problem, based on the priority value of the tags attached to the problem and time since it were committed. \fixme{Nogen der kan huske om vi stadig kigger p\aa{} tid siden det blev committed?}

%\paragraph{Add \client[]} adds a \client[] to the model. 

%\paragraph{Remove \client[]} removes a specific \client[] from the model. 

\paragraph{Set role} This is a trivial function, which grants a specific persons role.
\end{comment}