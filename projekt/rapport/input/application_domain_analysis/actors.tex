The system has two primary actors \client{} and \staff. Which are described in details in figure \ref{fig:actorclient} and \ref{fig:actorstaff}. Beside these two the system has the two actors: \sadmin{} and the non-human actor \wmon. The \sadmin{} can be a regular \staff{} with more privileges or a manager of the \staff. The \sadmin{} can change the tags and categories, create/remove departments and add/hire/remove \staff{} to the departments. The \wmon[] is an automated system which on a given time interval checks if some \staff{} members is overload with tasks(problems). In that case it will reassign the problems.

\begin{sadlist}{\Client}{Description of the actor \client.}{fig:actorclient}
\sadb{Goal:} A person who has a problem, and his goal is to get his problem(s) solved.

\sadb{Characteristics:} The \client{}s are employees or students with different knowledge and experience with similar systems. The clients prefer different ways of communication.%Skal nok overvejes.

\sadb{Examples:} \Client{} A prefers face-to-face communication with the working \staff{} whenever he got a problem. 

\client[c] B prefers web/mail communication as s substitution of face-to-face communication so he does not have to leave his working space in order to get help. 

\end{sadlist}




\begin{figure}[htps]

\begin{sadlistar}{\Staff}

\sadb{Goal:} The \staff{} solves the \client[]s problems and use the system as a taskmanager.  

\sadb{Charcteristics:} The \staff[] are employees and has various levels sophistication.

\sadb{Examples:} \staff[c] A prefers to speak to his manager and the \client[] face-to-face.

\staff[c] B enjoy getting his daily tasks from a computer system, 
 \end{sadlistar}
 \caption{\myCaption{Description of the actor \staff.}}
 \label{fig:actorstaff}
 \end{figure}