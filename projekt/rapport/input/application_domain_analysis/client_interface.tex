\subsection{Client Interface}
\label{sec:client_interface}

After login, the \aclient[] is greeted with the \textit{main} screen, wich gives three posebilities:
\begin{itemize}
	\item Add problem
	\item My problems
	\item Search for problem(s)
\end{itemize}

\paragraph{Add problem} as the name dictates, initiates the process of adding a specific problem to the system. Process starts by selecting what kind of problem the \aclient[] has. This is done by selecting \textit{tags} which describes the problem. Tags are grouped under categories. Each tag can only exist under one category, however if the need arises, it is possible to create a duplicate tag under another category. This does not mean that the same tag exists under two categories but two different tags with the same name exists under two different categories. We say that the problem is ``categorized'' when it has tags associated with it.\\
When the \aclient[] is satisfied with the categorization of his/hers problem, he/she can click ``Search'', which will make the system search for problems with similair categorization, prioritizing those with a solution attached to it.\\
From there, the idea is that the \aclient[] might find a problem which is identical, or almost identical, which solution will also fix the \aclient ´s problem. If such a problem is found, the \aclient[] has the option to \textit{``subcribe''} to the problem, making the \aclient[] receive all the same notifications as the person who created the problem, of which the \aclient[] just subscribed to.
\paragraph{My problems}
\paragraph{Search for problem(s)}



\begin{figure}[h]
\begin{center}
 \includegraphics[scale=0.70]{input/application_domain_analysis/client_interface}
\caption{\Client[] interface}
\label{fig:client_interface}
\end{center}
\end{figure}

