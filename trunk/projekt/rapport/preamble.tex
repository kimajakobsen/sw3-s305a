\documentclass[a4paper]{book}


% tilføjet af Stubman 231010
\usepackage[footnote,draft,danish,silent,nomargin]{fixme}		% Indst rettelser og lignende med \fixme{...} Med final i stedet for draft, udlses en error 																															for hver fixme, der ikke er slettet, nr rapporten bygges.

%%%%%%%%%% Various Packages %%%%%%%%%
\usepackage[english]{babel}
\usepackage[latin1]{inputenc}
\usepackage[numbers]{natbib}
\usepackage{verbatim}%% For Comment enviroment
%\usepackage{listings}
\usepackage{lastpage}% gives lastpage commando
\usepackage{algorithm}%%% Algorithm Eviroment
\usepackage{algorithmic}%%% Algorithm Eviroment
\usepackage{amsmath, amsfonts, amssymb, amsthm} % Math Paths
\usepackage{fancyhdr} % For headers
\usepackage{sadlist} % Selfdefined package. Gives \begin{sadlist}{title}{description}{label}
\usepackage{casecontrol}%
\usepackage{xifthen}% provides \isempty test and ifthen else 
\usepackage[absolute]{textpos} %used on the frontpage for the picture.
\usepackage{tabularx}
%%%%%%%%%%%%%%%%%%%%%%%%%%%%%%

%%%%%%% Protect against orhpans and widows %%%%%%%%%
\widowpenalty=300
\clubpenalty=300
%%%%%%%%%%%%%%%%%%%%%%%%%%%%%%%



%%%%%% Depracted  %%%%%%%%
%\usepackage{en-bib}  % anyone who knows this?
%\usepackage{en-bib}  % anyone who knows this?
%%%%%%%%%%%%%%%%%%%%


%%%%%%%%% Make links work %%%%%%%%%%
\usepackage[pdfborder={0 0 0 0}, backref=none]{hyperref}
%\usepackage[a4paper, bookmarks=true, bookmarksopen=true, bookmarksnumbered=true, pdfborder={0 0 0 0}, colorlinks=true, breaklinks=true, backref=section]{hyperref}
\hypersetup{
pdfborder=0 0 0
}
%%%%%%%%%%%%%%%%%%%%%%%%%%%%%%


%%% something? useful? hopefully!
%\usepackage[]{graphicx} %dvips
\usepackage{emp}
\ifx\pdftexversion\undefined
\usepackage[dvips]{graphicx}
\else
\usepackage[pdftex]{graphicx}
\DeclareGraphicsRule{*}{mps}{*}{}
\fi
\usepackage[]{subfig}% Need to make several pictures in one float
\usepackage{wrapfig}% Enables us to wrap text around a figure
%%%%%%%%%%%%%%%%%%%%%%%%%%%%%


%%%%%% Prettier chapters %%%%%
%\usepackage[Lenny]{fncychap}%
%\usepackage[Sonny]{fncychap}%
\usepackage[Bjornstrup]{fncychap}%
%\usepackage[Conny]{fncychap}%
%\usepackage[Rejne]{fncychap}%
%\usepackage{xparse}%
%%%%%%%%%%%%%%%%%%%%



%\setcitestyle{numbers}

%%%% Bibliography %%%%%%
\bibliographystyle{plainnat}
%%%%%%%%%%%%%%%%%


%%%%%% Something something might be important  %%%%%%%%%
\setcounter{tocdepth}{2}
\linespread{1}
%%%%%%%%%%%%%%%%%%%%

%%%%%%%%% Depracted %%%%%%%%%%%
%\setlength{\marginparwidth}{10pt}
%\setlength{\textwidth}{400pt}
%\setlength{\textheight}{620pt}
%\setlength{\voffset}{0pt}
%\setlength{\hoffset}{0pt}
%\setlength{\topmargin}{0pt}
%\setlength{\headsep}{10pt}
%\setlength{\oddsidemargin}{50pt}
%\setlength{\evensidemargin}{10pt}
%%%%%%%%%%%%%%%%%%%%



%%%%%%%%%%%%  COMMANDS   %%%%%%%%%%%%%%%%%%%%%%%
%%%%%%%%%%%%%%%%%%%%%%%%%%%%%%%%%%%%%%%%%%%
%%%%%%%%%%%%%%%%%%%%%%%%%%%%%%%%%%%%%%%%%%%
%%%%%%%%%%%%%%%%%%%%%%%%%%%%%%%%%%%%%%%%%%%

%%%%%%%%%% Get commands defined Elsewhere %%%%%%%%%
%\input{functions/caseControl}
\input{functions/myDate}
%% This is where we define words. 
%% The last parameter should be blank as standard, but you can add i,b,u. 

%% WORD LIST
%% This is where we define words. 
%% The last parameter should be blank as standard, but you can add i,b,u. 

\newcommand{\ainterface}[1][]{\caseControl{A}{dmin Interface}{#1}{}}%
\newcommand{\cinterface}[1][]{\caseControl{C}{lient Interface}{#1}{}}%
\newcommand{\sinterface}[1][]{\caseControl{S}{taff Interface}{#1}{}}%
\newcommand{\helpdesk}[1][]{\caseControl{H}{elpdesk}{#1}{}}%
\newcommand{\hdesk}[1][]{\caseControl{H}{elpdesk}{#1}{}}
\newcommand{\wmon}[1][]{\caseControl{w}{orkload monitor}{#1}{}}
\newcommand{\staff}[1][]{\caseControl{s}{taff}{#1}{}}
\newcommand{\client}[1][]{\caseControl{c}{lient}{#1}{}{}}%
\newcommand{\astaff}[1][]{\caseControl{s}{taff}{#1}{}}
\newcommand{\aclient}[1][]{\caseControl{c}{lient}{#1}{}{}}%
\newcommand{\sadmin}[1][]{\caseControl{a}{dmin}{#1}{}}
\newcommand{\admin}[1][]{\caseControl{a}{dmin}{#1}{}}



\newcommand{\ucsproblem}[1][]{\caseControl{s}{ubmit problem}{#1}{}}
\newcommand{\ucsolproblem}[1][]{\caseControl{s}{olve problem}{#1}{}}
\newcommand{\closed}[1][]{\caseControl{c}{losed}{#1}{}}
\newcommand{\open}[1][]{\caseControl{a}{ctive}{#1}{}}
\newcommand{\solved}[1][]{\caseControl{c}{losed}{#1}{}}
\newcommand{\unsolved}[1][]{\caseControl{a}{ctive}{#1}{}}
\newcommand{\pinactive}[1][]{\caseControl{c}{losed}{#1}{}}
\newcommand{\pactive}[1][]{\caseControl{a}{ctive}{#1}{}}
\newcommand{\gstat}[1][]{\caseControl{g}{et statistics}{#1}{}}
\newcommand{\bloadwork}[1][]{\caseControl{b}{alance loadwork}{#1}{}}
\newcommand{\worklist}[1][]{\caseControl{w}{orklist}{#1}{}}
\newcommand{\todolist}[1][]{\caseControl{w}{orklist}{#1}{}}
\newcommand{\tucadmin}[1][]{\caseControl{a}{dministrate}{#1}{}}

\newcommand{\Hdesk}[1][]{\hdesk[c]}
\newcommand{\Wmon}[1][]{\wmon[c]}
\newcommand{\Staff}[1][]{\staff[c]}
\newcommand{\Client}[1][]{\client[c]}
\newcommand{\Astaff}[1][]{\astaff[c]}
\newcommand{\Aclient}[1][]{\aclient[c]}
\newcommand{\Helpdesk}[1][]{\helpdesk[c]}
\newcommand{\Sadmin}[1][]{\sadmin[c]}
\newcommand{\Admin}[1][]{\admin[c]}


\newcommand{\john}[1][]{\caseControl{j}{ohn}{#1}{}}
\newcommand{\michael}[1][]{\caseControl{M}{ikael}{#1}{}}
\newcommand{\rubik}[1][]{\caseControl{R}{ubik's Cube}{#1}{}}
\newcommand{\facet}[1][]{\caseControl{f}{acelet}{#1}{}}
\newcommand{\facelet}[1][]{\caseControl{f}{acelet}{#1}{}}
\newcommand{\cube}[1][]{\caseControl{R}{ubik's Cube}{#1}{}}
\newcommand{\cuber}[1][]{\caseControl{c}{uber}{#1}{}}
\newcommand{\face}[1][]{\caseControl{f}{ace}{#1}{}}
\newcommand{\cpiece}[1][]{\caseControl{c}{ubie}{#1}{}}% Not that the one below are the same as this one
\newcommand{\cubie}[1][]{\caseControl{c}{ubie}{#1}{}}%
\newcommand{\cubicle}[1][]{\caseControl{c}{ubicle}{#1}{}}
\newcommand{\twist}[1][]{\caseControl{t}{wist}{#1}{}}
\newcommand{\turn}[1][]{\caseControl{t}{urn}{#1}{}}
\newcommand{\rotate}[1][]{\caseControl{r}{otate}{#1}{}}
\newcommand{\erno}[1][]{\caseControl{E}{rn\"{o} Rubik}{#1}{}}
\newcommand{\mpuzzle}[1][]{\caseControl{M}{agic Puzzle}{#1}{}}
\newcommand{\msquare}[1][]{\caseControl{M}{agic Square}{#1}{}}
\newcommand{\mcube}[1][]{\caseControl{M}{agic Cube}{#1}{}}

 
%%%%%%%%%%%%%%%%%%%%

%%%%  Makes the titles look nicer. i guess. Rasmus out. %%%%%%% Don't remove
%\usepackage{titlesec} \newcommand{\bigrule}{\titlerule[0.5mm]} \titleformat{\chapter}[display] {\bfseries\Huge} {  \vskip-2em 
 %\titlerule 
% \filright  \huge\chaptertitlename\ \vspace{5mm}  \huge\thechapter} {0mm} {\filright} [\vspace{3mm} \bigrule \vspace{-10mm}] %
%%%%%%%%%%%%%%%%%%%%%%%%%%%%

%%%%% Depracted %%%%%%%%%
\newcommand{\picturepath}[1]{input/pics/}


%%%%% Used to determine the highlight of the first word in the terminology %%%%%
\newcommand{\myTermHigh}[1]{\textbf{#1}: }
%%%%%%%%%%%%%%%%%%%%%%

%%%%%%%%% tops 'n' tails %%%%%%%%%%%%
\newcommand{\myTop}[1]{\vspace{-8mm}  \vspace{0mm} \textit{#1} \vspace{3.4mm} \hrule \vspace{3.4mm} }
%\newcommand{\myTop}[1]{ \vspace{3.4mm} \textit{#1} \  \\  \hrule \  \\}%
\newcommand{\myTail}[1]{ \vspace{3.4mm} \hrule \vspace{3.4mm} \textit{#1} }%

\newcommand{\emptyTop}[0]{\vspace{-6mm}}
%%%%%%%%%%%%%%%%%%%%%%%%%%%%

%%%%%%%%% Use this for caption text %%%%%%%%%%
\newcommand{\myCaption}[1]{\textit{\footnotesize #1.}}
\newcommand{\morscaption}[1]{\caption{\myCaption{#1}}}
%\usepackage[bf]{caption}
%%%%%%%%%%%%%%%%%%%%%%%%%%%%

%%%%%%%% Bruges til skillekolonner og r\ae{}kker . Definerer tykkelsen. %%%%%%
\newcommand{\vrules}{{\vrule width 0.6pt}}
\newcommand{\hrules}{{\hrule height 1.2pt}}
%%%%%%%%%%%%%%%%%%%%%%%%%%%%%%%%%%%%%%%%%%%

%%%%%%%Commands for getting e.g. ``st'' lifted in 1st   %%%%%%%%%%%
\newcommand{\superscript}[1]{\ensuremath{^{\textrm{#1}}}}
\newcommand{\subscript}[1]{\ensuremath{_{\textrm{#1}}}}
\newcommand{\ths}[0]{\superscript{th}}
\newcommand{\st}[0]{\superscript{st}}
\newcommand{\nd}[0]{\superscript{nd}}
\newcommand{\rd}[0]{\superscript{rd}}
%%%%%%%%%%%%%%%%%%%%%%%%%%%%%%%%%%%%%%%%%%%


%%%%%%%%%% Moves %%%%%%%%%%
\newcommand{\m}[1]{\textbf{#1}}
\newcommand{\vr}[1]{$#1$}
\newcommand{\pcparagraph}[1]{\vspace{-1.5mm}\paragraph{#1}}
%%%%%%%%%%%%


%%%%%%%%%% Class, component ... %%%%%%%%%%%%%%%%%%

\newcommand{\me}[1]{\textit{#1}}

\newcommand{\class}[1]{\textbf{#1}}						%%Do change
\newcommand{\cl}[1]{\class{#1}}								%%Don't change
\newcommand{\aclass}[1]{\textit{\class{#1}}}	%%Do change
\newcommand{\acl}[1]{\aclass{#1}}							%%Don't change
\newcommand{\component}[1]{\textbt{#1}}				%%Do change
\newcommand{\comp}[1]{\component{#1}}					%%Don't change
\newcommand{\vari}[1]{\textbf{#1}}					%%Don't change
%%%%%%%%%%%%%%%%%%%%%%%%%%%%%%%%%%%%%%%%%%%%%%%%%%

%%%%%%%%%%%%%%% Qoutations %%%%%%%%%%%%
\newcolumntype{R}{>{\raggedleft\arraybackslash}X}%
\newcommand{\myQuote}[1]{
\begin{flushleft}
\Huge{\textbf{``}}
\end{flushleft}
\vspace{-30pt}
\begin{quote}
#1
\end{quote}
\vspace{-30pt}
\begin{flushright}
\Huge{\textbf{''}}
\end{flushright}
}


%%%%%%%%% Defining Theorem %%%%%%%%%%%
\theoremstyle{definition} \newtheorem{theorem}{Theorem}
%%%%%%%%%%%%%%%%%%%%%%%%%%%%%%%%%%%%%%%%%%%

%%%%%%%%% Multi Row %%%%%%%%%%%%%%%%%%%
\usepackage{multirow}
%%% This is needed to use the multicolumn command
%%%%%%%%%%%%%%%%%%%%%%%%%%%%%%%%%%%%%%%%%%%
\hyphenation{help-desk}

%%%%%%%%% List Environments %%%%%%%%%%%
\usepackage{listings}


\usepackage{color}

\definecolor{gray95}{gray}{.95}
\definecolor{gray92}{gray}{.92}
\definecolor{gray75}{gray}{.75}
\definecolor{gray45}{gray}{.45}

\lstdefinestyle{sourceCode}
{ 
	numbers=left,
	numbersep=5pt, 
	stepnumber=1,
	captionpos=b,  %bottom
	keywordstyle=\color[rgb]{0,0,1},
	commentstyle=\color[rgb]{0.133,0.545,0.133},
	stringstyle=\color[rgb]{0.627,0.126,0.941},
	backgroundcolor=\color{gray95},
	frame=lrtb,
	framerule=0.5pt,
	linewidth=1.00\textwidth,
	tabsize=4,
	numberbychapter=true,
	basicstyle=\ttfamily\footnotesize,
	language=[Sharp]C,
	breaklines=true,
	showstringspaces=false,
	emph=[1]{value,endregion,region,get,set},%%%%%%%%%%% Add new keywords here
	emph=[2]{Tag,Problem,Person,List,NotSupportedException,TestMethod,ProblemSearch,Assert,
	EntityCollection,Department,IEnumerable,TimeSpan,DateTime},%%Classes
	emphstyle=[1]{\color[rgb]{0,0,1}},
	emphstyle=[2]{\color[rgb]{0.1,0.5,0.5}},
	float=htb,
	breakindent=20pt
}


\renewcommand{\lstlistingname}{Code snippet}%%Changing the caption to read ``Code snippet''
\renewcommand{\lstlistlistingname}{List of Code Snippets}
\lstset{escapeinside={(*}{*)}}%%Defines the escape and unescape chars

%%%%%%%%%% To use with copy-paste:
\begin{comment}

\begin{lstlisting}[style=sourceCode, caption=\myCaption{<some caption>}, label=<some label>]
<the code>
	<more code, now with indent>
\end{lstlisting}

\end{comment}
%%%%%%%%%% To input file:
\begin{comment}

\lstinputlisting[style=sourceCode, caption=\myCaption{<some caption>}, label=<some label>]{<file name>}

\end{comment}


