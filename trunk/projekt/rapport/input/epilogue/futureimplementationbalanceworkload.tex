\section{BalanceWorkload}
\label{sec:futureimplementationbalanceworkload}

The original \me{BalanceWorkload} method distributes the problems by looking at the staff member which has the highest workload as well as the one with the lowest, and then shifts the reassignable problems until the workload between the two are as equal as possible. This approach doesn't take the priority or whether or not the deadline is approved into account.

The idea behind the \me{FutureImplementationBalanceWorkload} method -- which is seen in code snippet \ref{lst:futureimplementationbalanceworkload} -- is that the algorithm itself doesn't have to balance the workload, other than always choosing the person with the smallest workload, whenever a problem is about to be distributed. 

\begin{lstlisting}[style=sourceCode, caption=\myCaption{A possible future implementation of the \me{BalanceWorkload} algorithm.}, label=lst:futureimplementationbalanceworkload]
public void FutureImplementationBalanceWorkload()
{
    Person dummyPerson = new Person();
    List<Person> staffMembers = Persons.ToList();
    List<Problem> problemList = new List<Problem>();

    foreach (var member in staffMembers)
    {
        foreach (var problem in member.Worklist)
        {
            problemList.Add(problem);
        }
    }

    problemList = problemList.Where(x => x.Reassignable == true && x.SolvedAtTime == null).ToList();

    foreach (var problem in problemList)
    {
        problem.AssignedTo = dummyPerson;
    }

    while (problemList.Count > 0)
    {
        // find the person with the lowest workload
        Person min = Persons.FirstOrDefault(y => y.GetWorkload() == Persons.Min(x => x.GetWorkload()));

        // assign the most important problem to the person
        problemList[0].AssignedTo = min;
        problemList.RemoveAt(0);
    }
}
\end{lstlisting}
