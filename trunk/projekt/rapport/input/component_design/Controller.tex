\subsection{Controller}

The controller classes are used to display the model functionality and feed the model with data.

\begin{description}
\item[Account]\hfill
\begin{description}
\item[Purpose:]Register new users, manage the user login and authorize user roles.
\item[Attributes:]Username, password, email.
\item[Operations:]Register users, user login, authorize roles.
\end{description}
\end{description}

\begin{description}
\item[Person]\hfill
\begin{description}
\item[Purpose:]Manage client, staff and admin functions.
\item[Attributes:]Username, problem id, role, description, person id, department id, 
\item[Operations:]View problem, search problems, attach solution, list solutions, detach solution, add solution, edit, create, change department, edit person, choose department, delete person, add user to role, remove user from role, mail.
\end{description}
\end{description}

\begin{description}
\item[Department]\hfill
\begin{description}
\item[Purpose:]Create and manage departments.
\item[Attributes:]Department name, department id.
\item[Operations:]Create department, edit departmentm delete department.
\end{description}
\end{description}

\begin{description}
\item[Category]\hfill
\begin{description}
\item[Purpose:]Create new categories and manage categories as well as aggregate categories and tags.
\item[Attributes:]Category name, category id.
\item[Operations:]Create category, edit category, delete category, hide/unhide categories, hide/unhide tags.
\end{description}
\end{description}

\begin{description}
\item[Tag]\hfill
\begin{description}
\item[Purpose:]Create, delete and manage tags.
\item[Attributes:]Tag name, tag id.
\item[Operations:]create, edit, delete.
\end{description}
\end{description}

\begin{description}
\item[Problem]\hfill
\begin{description}
\item[Purpose:]Create, delete and manage problems.
\item[Attributes:]Problem id, problem name, person id, date, person name, comment, staff name, problem description.
\item[Operations:]Categorize new problem, similar problem, create, subscribe, unsubscribe, add comment.
\end{description}
\end{description}

\begin{description}
\item[Home]\hfill
\begin{description}
\item[Purpose:]Create client, staff and admin roles if the helpdesk don't contrain any roles. If there is no users in the application or no users are admin the home class will create a root user.
\item[Attributes:]Username, password, email, role name.
\item[Operations:]Create client, staff and admin roles, create root user.
\end{description}
\end{description}

\begin{description}
\item[Reassign]\hfill
\begin{description}
\item[Purpose:]Assign problems to staff members and reassign problems to staff members on request.
\item[Attributes:]Staff id, problem id.
\item[Operations:]Assign problem, reassign problems.
\end{description}
\end{description}

\begin{description}
\item[Statistics]\hfill
\begin{description}
\item[Purpose:]Calculate statistics for departments and staff members.
\item[Attributes:] Staff id, department id, total time taken to solve all problems, number of problems.
\item[Operations:]Calculate statistics for departments, calcuate statistics for staff members.
\end{description}
\end{description}