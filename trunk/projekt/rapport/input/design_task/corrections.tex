\section{Corrections to the Analysis}
\label{sec:correctionstotheanalysis}

We reviewed the analysis and came up with some corrections. Subsection \ref{}, \ref{}, and \ref{} contains the corrections, modifications, and supplement to the analysis. 

\subsection{Classes and events}


\subsection{}


\subsection{}


The altered version of our class diagram -- which can be seen on figure \ref{fig:ClassDiagramV2} -- contains the following changes:
\begin{itemize}	
	\item The role pattern: \\
	The functionality from the actor \admin[] does not inherit from \astaff[] and \aclient[] etc. Figure \ref{tab:newactortable} shows the new role system.   
	\item Problem: 
	\begin{itemize}
		\item \textbf{Deadline} \\
					The deadline is now a attribute instead of a class
		\item \textbf{Problem and tag relation} \\
					A tag can be connected to multiple different problems and a problem can have many tags connected to it. 
		\item \textbf{Problem and person relations} \\
					A person can have from zero to three simultaneous roles.				
	\end{itemize}
\end{itemize}

\begin{figure}[p]
\begin{center}
\begin{tabular}{l  ccccc}
\hline 
\multicolumn{2}{r}{\shf{Actor}} \\
\shf{Use case} 	&   \Aclient 	& \Astaff 		& \admin[c]  \\ \hline%
Submit problem 	& $\checkmark$ 	&  	&  \\ %
My problems 		& $\checkmark$	&   &  \\ %
Worklist 				& 	& $\checkmark$  &  \\ %
Solve problem 	& 	& $\checkmark$	&  \\ %
Administrate		&  	&		& $\checkmark$ \\	%
\gstat[c]				&		& 	& $\checkmark$ \\ \hline%
\end{tabular}
\end{center}
\caption{\myCaption{Actor \& use case table}}
\label{tab:newactortable}
\end{figure}