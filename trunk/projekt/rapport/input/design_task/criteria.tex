\section{Quality Goals}
\label{sec:criteria}
The design of \hdesk[] is specified from the following quality goals: Usable, secure, efficient, consistence, reliable, maintainable, testable, comprehensible, reusable, portable, and interoperable.
The definition of these quality goals -- or simply criteria -- is shown in figure \ref{fig:defOfCrit}.
\cite[p.~178]{roedeaalborg}

\begin{sable}[h]{0.3}{0.7}{Definition of criteria}{fig:defOfCrit}
 \shfone{Criterion}&\shftwo{Definition} \\
\hline \\
  Usable&The end user can easily use the application \\ \\
  Secure&Precautions against unauthorized access \\ \\
  Efficient&How well the resources available are being used \\ \\
  Consistence&How correct the data in the model is \\ \\
  Reliable&The degree of the applications accessibility \\ \\
  Maintainable&The cost of locating and fixing application errors \\ \\
  Testable&The cost to ensure that the application performs intentionally \\ \\	
	Flexible&How easily the application can be setup to fit the structure of an institution. \\ \\  %Flexible&The cost for the end user to modify the system after deployment \\ \\
	%Comprehensible&How easy it is for the end user to understand the application \\ \\
	Comprehensible&How easy it is to understand the application \\ \\

  Reusable&The potential for using parts of this application in another application \\ \\
  Portable&How much effort needed to change the platform of the application \\ \\
  Interoperable&How well the application cooperates with other applications \\
\end{sable}

%\subsection{Quality Goals}
%Since it is not possible to prioritize each criteria equally,
We have chosen to use a four step priority scale: Very important, important, less important, and irrelevant.
Each criteria defined in figure \ref{fig:defOfCrit} is prioritized in figure \ref{fig:prioritizedCrit}. Furthermore the figure shows a column labeled ``Easily Fulfilled'' which specifies whether a given criteria will be fulfilled without much effort.
%We would like to explicity state, that these quality goals are not about what's optimal for the application as a product, which should actually be installed and used, but instead as what we have focused on during our development.

\begin{figure}[h]
	\centering
		\begin{tabular}{| l | m{0.135\textwidth} | m{0.135\textwidth}| m{0.13\textwidth}| m{0.135\textwidth}|m{0.13\textwidth}|} \hline
												& Very  Important 												& Important 														& Less Important 											& Irrelevant 												& Easily Fulfilled 									\\ \hline
		Usable  						& 																				& 												& \multicolumn{1}{c|}{$\checkmark$}		& 																	& 																	\\ \hline
		Secure  						& 																				& 																			& \multicolumn{1}{c|}{$\checkmark$} 	& 																	& 																	\\ \hline
		Efficient 					& 																				& 																			& 																		& \multicolumn{1}{c|}{$\checkmark$} & 																	\\ \hline
		Consistence  				& 																				& \multicolumn{1}{c|}{$\checkmark$} 		& 																		& 																	& 																	\\ \hline
		Reliable  					& 																				& 																			& \multicolumn{1}{c|}{$\checkmark$} 	& 																	& 																	\\ \hline
		Maintainable  			& 																				& 																			& 																		& \multicolumn{1}{c|}{$\checkmark$} & 																	\\ \hline
		Testable  					& 																				& \multicolumn{1}{c|}{$\checkmark$} 		& 																		& 																	& 																	\\ \hline
		Flexible  					& \multicolumn{1}{c|}{$\checkmark$} 			& 																			& 																		& 																	& 																	\\ \hline
		Comprehensible  		& 																				& \multicolumn{1}{c|}{$\checkmark$} 		& 																		& 																	& 																	\\ \hline
		Reusable  					& 																				& 																			& 																		& \multicolumn{1}{c|}{$\checkmark$} & 																	\\ \hline
		Portable 						& 																				& 																			& 																		& \multicolumn{1}{c|}{$\checkmark$} &	\multicolumn{1}{c|}{$\checkmark$} \\ \hline
		Interoperable 			& 																				& 																			& \multicolumn{1}{c|}{$\checkmark$} 	& 																	& 																	\\ \hline
		\end{tabular}
	\morscaption{The criteria with a priority}
	\label{fig:prioritizedCrit}
\end{figure}

How we have prioritized the criteria is based on our application definition, which is found in section \ref{sec:systemdefinition}. The reasoning for the priority of each criteria in figure \ref{fig:prioritizedCrit} is shown bellow.

\paragraph{Usable}

It is important that our application is user friendly because a helpdesk needs to be very usable to ease the process of getting help and the end user will not use the helpdesk if it is unusable.
it is however not very important since this is a study project and we are more concerned with the functionality of the application than the usability, hence; less important 
\paragraph{Secure}
For the \hdesk[] security is less important. We want to distinguish between \aclient[]s, \astaff[] members and \admin[]s -- the clients are e.g. not allowed to solve problems or choose who should be assigned to what problem.
We do not have any sensitive information, other than e-mail addresses and passwords. We do not take any direct measures to prevent data interception or any other serious security flaw. We do, however encrypt the passwords which are stored in the database.
We do however not have any sensitive information, so we take no measures to prevent data interception or any other serious security flaw.
\paragraph{Efficient}
We do not focus on the efficiency of our application, but only that it works which lead us to irrelevant for this criterion.
\paragraph{Consistence}
It is important that end users can see which problems are solved and which are not.
Furthermore the model should not contain duplicates, since it could compromise the integrity of the statistics which the application generates.
This lead us to prioritize consistence as important.
\paragraph{Reliable}
The reliability of the \hdesk[] application is not of great interest to us.
%Of course we do not want to sabotage our system to malfunction at any time, we just want to pay our attention to other criteria instead.
We do not actively do anything to increase the reliability of the application, neither do we intensionally decrease it.
We want to pay our attention to other criteria instead, therefore this criterion is prioritized less important.
\paragraph{Maintainable}
Since we do not intend to maintain the application after it is finished, it is prioritized as irrelevant.
\paragraph{Testable}
We want our application to work and to make sure it does, we will run tests.
Therefore we want our application to be testable, hence testable is important. 
\paragraph{Flexible}
It is very central to our application that it is flexible, because we want it to be generic -- that it can be adapted to any organization without much or any cost. To insure this we added features to manage departments, manage persons, manage categories, and manage tags at runtime. We have chosen this to be a very important criteria.
\paragraph{Comprehensible}
Since this is study project it is important that our application is comprehensible in order to explain how it works. 
%Since the \hdesk[] application is supposed to be generic, it should be easy for the user to understand it.
%However our focus is primarily on functionality, so we have prioritized it important.
\paragraph{Reusable}
Since we do not care much for the application after it is deployed, we do not care whether or not it is reusable, hence irrelevant.
\paragraph{Portable}
Our applications portability can be divided into two, the client side and the server side.
We do not care about the portability of the server side, because we will rather focus on the portability on the client side and the functionality of the application.
Therefore it is considered irrelevant.
The end users will access our application through a browser, so we assume that it can be easily fulfilled since there are many browsers for different platforms. \cite{chrome10}\cite{firefox}
\paragraph{Interoperable}
If it is possible we want to be able to use an existing database for authentication to our application.
This is however the only other application which we plan on cooperating with, therefore it is prioritized less important.