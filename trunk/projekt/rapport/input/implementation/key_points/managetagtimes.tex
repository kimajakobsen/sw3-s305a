\section{Tag Time Management}
\label{sec:managetagtimes}




\begin{lstlisting}[style=sourceCode, caption=\myCaption{The \me{ManageTagTimes} method}, label=lst:managetagtimes,float=hp]
public void ManageTagTimes(double StaffTimeSpentInput)
{
	int MinutesUsed = (int)(StaffTimeSpentInput*60);
	
	foreach (var tag in Tags)
  {
		if (tag.SolvedProblems == null)
		tag.SolvedProblems = 0;
    
		tag.SolvedProblems++;

		if (tag.TimeConsumed == null)
			tag.TimeConsumed = 0;

		tag.TimeConsumed = tag.TimeConsumed + MinutesUsed;
	}
}
\end{lstlisting}


When a problem is solved the \astaff[] member who solved it sets an estimate of how long he/she used to solve the problem, this estimate is saved in the tags of the problem and later used to calculate statistics of problems with these tags. 
The \me{ManageTagTimes} method updates the properties \vari{SolvedProblems} and \vari{TimeConsumed} of the class \cl{Tag}.
By incrementing the \vari{SolvedProblems} and \vari{TimeConsumed} properties of each tag respectively by one and the time used for the problem to be solved.
The method is shown in code snippet \ref{lst:managetagtimes}.



%updates the statistics held by each tag when a staff member marks a problem as solved and submits the number of hours he/she used solving it. The \me{ManageTagTimes} method is a property of the \cl{Problem} class in the model.
%The idea behind it is simply to increment the \verb+SolvedProblems+ and \verb+TimeConsumed+ properties of each tag. The method is shown in code snippet \ref{lst:managetagtimes}.