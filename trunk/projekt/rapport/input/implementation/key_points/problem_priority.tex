\section{Problem Priority}
\label{sec:problem_priority}

When assigning multiple problems to a staff member, the'll appear as a list. Due to the human nature, we might pick a specific order of solving the problems in, which doesn't necessarily take priority, deadlines and such into account. Therefore, the order of which the problems appear in in the staff members worklist, is important.

We've defiend two elements which define a problems importance, and therefore it's placement in the list of problems. These two elements are:

\begin{itemize}
	\item whether or not a deadline is set
	\item the priority
\end{itemize}

We acknowledge that problems which has a deadline should be at least reviewed by the assigned staff member earlier than problems without a deadline.

The list is ordered by priority of the problem, however problems with approved deadlines will always appear on top of the list regardless of their or other problems priority. This splits the list two parts. Above, priority-sorted problems with approved deadlines, and below priority-sorted problems with or without not approved deadlines. 

If a problem with a deadline is overdue, then the priority will go up to the maximum value 10, which in turn will make the problem appear on the absolute top of the list.

An example can be seen in figure \fixme{Inds\ae t figur fra ``Program presentation'' section der viser det.} 