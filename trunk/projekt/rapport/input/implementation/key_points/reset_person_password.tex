\section{Reset Person Password}
\label{sec:resetpersonpassword}

\begin{lstlisting}[style=sourceCode, caption=\myCaption{The ResetPassword(String user) method}, label=lst:resetpersonpassword]
public static String ResetPassword(String user)
{   
    //Connection
    SqlConnection cn = new SqlConnection();
    cn.ConnectionString = connString;
    //Preparing SqlCommand
    SqlCommand password;
    SqlCommand userId;

    //SqlCommand to find UserId
    userId = new SqlCommand("SELECT UserId FROM aspnet_Users WHERE (UserName = '" + user + "')", cn);
    
    //Executing the SqlCommand
    cn.Open();
    String userA = userId.ExecuteScalar().ToString();
    password = new SqlCommand("SELECT Password FROM aspnet_Membership WHERE (UserId = '" + userA + "')", cn);
    String passA = password.ExecuteScalar().ToString();
    //Converting Application into a String
    cn.Close();

    String[] passarray =
    {
        "a","b","c","d","e","f","g","h","j", "i" ,"k",
        "l","m","n","o","p","q","r","s","t","u",
        "w","x","y","z","A","B","C","D","E","F",
        "G","H","I","J","K","L","M","N","O","P","Q",
        "R","S","T","U","V","W","X","Y","Z",
        "0","1","2","3","4","5","6","7","8","9"
    };

    String setPass = "";
    //Preparing a two randoms, one for password length and one for picking a char from passarray.
    Random RandomNumber = new Random();
    Random RandomPass = new Random();
    int x = RandomNumber.Next(10,25);
    //Putting the passowrd together
    for (int i = 0; i < x; i++)
    {
        int y = RandomPass.Next(passarray.Length);
        setPass += passarray[y].ToString();
    }
    //Finding the user
    MembershipUser u = Membership.GetUser(user);
    //Generating a new password for the user
    String np = u.ResetPassword();
    //Changing the password for the user by using the the resetted password as old password
    u.ChangePassword(np, setPass);
    return setPass;
}
\end{lstlisting}