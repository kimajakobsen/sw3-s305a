\section{Problem Search}
The system needs to be able to search for problems.
This search is based on tags.
It will find an amount of problems which match the specified tags and order them by number of tags matching.
The amount of problems this function will find is depending on a specified number know as ``Minimum number of problems'', which determines when the function should stop searching for more problems.

The input for this function is:

\begin{itemize}
	\item Selected tags
	\item Problems to search among
	\item All tags
	\item Minimum number of problems to find
\end{itemize}

\begin{lstlisting}[style=sourceCode, caption=\myCaption{}, label=src:search]
while (result.Count < listMinSize && noOfTagsToRemove < tags.Count)
{
	tempResult = new List<Problem>();
	tagsToRemove = new List<int>();
	for (int i = 0; i < noOfTagsToRemove; i++)
	{
		tagsToRemove.Add(i);
	}
	try
	{
		List<Tag> currentSearch = tags.RemoveCurrent(tagsToRemove);
		while (true)
		{
			temp = allProblems.ToList();
			foreach (Tag tag in currentSearch)
			{
				temp = temp.Where(x => x.Tags.Contains(allTags.FirstOrDefault(y => y.Id == tag.Id))).ToList();
			}
			tempResult.AddRangeNoDuplicates(temp.ToList());
			currentSearch = tags.RemoveNext(ref tagsToRemove);
		}
	}
	catch (NotSupportedException)
	{
			noOfTagsToRemove++;
			tempResult.Sort(compare);
			result.AddRangeNoDuplicates(tempResult.ToList());
	}
}
\end{lstlistning}