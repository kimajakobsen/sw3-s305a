\section{Get statistics}
\label{sec:getstatistics}

The \cl{StatisticsController} controller -- seen in code snippet \ref{lst:getstatistics} -- calculates the statistics, by utilizing the \me{AveragePerProblem()} method, -- seen in code snippet \ref{lst:averageperproblem} --

%The \cl{StatisticsController} controller -- seen in code snippet \ref{lst:getstatistics} -- calculates the statistics, by utilizing the \me{AverageTimePerProblem()} method -- seen in code snippet \ref{lst:averagetimeperproblem} --  which utilizes the \me{AverageTimePerProblem()} method as well as the \me{AverageTimePerProblemLastWeek()} methods, which both are wrappers to the \me{AverageTime(string method)} method. All three is shown in code snippet \ref{lst:averagetime}.\\
The purpose is to view statistics 

\begin{lstlisting}[style=sourceCode, caption=\myCaption{The StatisticsController controller}, label=lst:getstatistics]
[Authorize(Roles = Constants.AdminRoleName)]
public class StatisticsController : Controller
{
    // GET: /Statistics/
    hoplaEntities db = new hoplaEntities();
    public ActionResult Index()
    {
        var departments = db.DepartmentSet.ToList();
        
        List<Problem> problems = new List<Problem>();
        List<Problem> problemsPastWeek = new List<Problem>();
        var now = DateTime.Now;
        var since = now.Subtract(new TimeSpan(7, 0, 0, 0));
        foreach(var dep in departments){
            foreach (var person in dep.Persons)
            {
                problemsPastWeek.AddRange(person.Worklist.Where(x => x.SolvedAtTime > since));
                problems.AddRange(person.Worklist.Where(x => x.SolvedAtTime != null));
            }
        }

        var viewModel = new StatisticViewModel()
        {
            AverageLastWeek = StatTool.AveragePerProblem(problemsPastWeek),
            AverageAllTime = StatTool.AveragePerProblem(problems),
            Departments = departments
            
        };
        return View(viewModel);
    }

}
\end{lstlisting}


\begin{lstlisting}[style=sourceCode, caption=\myCaption{The AveragePerProblem(IEnumerable<Problem> problems) method -- which is found in the \cl{StatTool} class.}, label=lst:averageperproblem]
public class StatTool
{
   
    public static TimeSpan AveragePerProblem(IEnumerable<Problem> problems)
    {
        if (problems == null || problems.Count() == 0)
            return new TimeSpan();
       int totalTime = 0; 
       int problemsCount = 0;
       foreach(var problem in problems){
           problemsCount++;
           totalTime = totalTime + (int)((TimeSpan)(problem.SolvedAtTime - problem.Added_date)).TotalMinutes;
       }

       return new TimeSpan(0,totalTime / problemsCount,0);
       
    }
}
\end{lstlisting}
