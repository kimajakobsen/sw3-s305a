\section{Development Tools}
In the creation of our web application we use some development tools which will be described in this section. 

\subsection{IDE}
\label{sub:ide}
As development tool we use Microsoft Visual Studio Ultimate \cite{visualStudio}, which include a large variety of inbuilt tools for unit testing, SQL and data modeling. 
We have experience with development using Visual Studio Ultimate from the Object-Oriented Programming course. 
Alternatively MonoDevelop \cite{mono} can be used -- which is an open-source cross platform .NET IDE -- but since none of us have experience with this tool, it is not preferred.
%We choose to use Visual Studio Ultimate since all the group members have experience with the IDE.

%http://www.microsoft.com/visualstudio/en-us/products/2010-editions/ultimate


\subsection{Collaboration}
\label{sub:collaboration}
For collaboration we use Subversion(SVN) and for the code sharing we use AnkhSVN \cite{ankhsvn} together with SVN. 
AnkhSVN is a source control provider for Microsoft Visual Studio. 
Alternatively we can use Team Foundation Server \cite{teamfoundation}, but this requires installation and configuration of a Team Foundation Server. 
We choose AnkhSVN since we already have a running SVN server. 

%ankhsvn  - http://ankhsvn.open.collab.net/
% Team foundation - http://msdn.microsoft.com/en-us/library/ms181232(VS.80).aspx
\begin{comment}
\subsection{Database}
\label{sub:database}
The choice of database vendor lies between two systems: Microsoft SQL Server and PostgreSQL. Microsoft because we already use Visual Studio and ASP.NET. PostgreSQL because we use this provider in the course Software Architecture. 
We choose Microsoft SQL Server as our main data storage because it is compatible with ADO.NET Entity Data Model Designer and Visual Studio Ultimate comes with an inbuilt Microsoft SQL Server manager, which allows editing of the SQL server from our workstations and not only from the server itself. 
Using PostgreSQL with C\# and Visual Studio requires a plugin in order to use the ADO.NET Entity Data Model Designer. Since the required functionality is supported by both database systems neither choice gives us any advantages over the other on the data layer.
\end{comment}