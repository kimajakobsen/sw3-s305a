\section{Behavior - Statecharts}
In this section of the report, we will describe the behavior of the most complex functions in our class diagram.
\fixme{(hvem skriver om det her?) - Det g\o r Lasse, og han vil pr\o ve at tage det med, s\aa snart han husker det.}

\subsection{Problem}
Figure \ref{fig:Klasse_diagram_problem} shows the behavior of a problem. Note that you can solve the problem by attaching one or solutions. A problem can have an unlimited number of solutions. You can at all times delete the problem, thus the arrow points from the edge of the box.
\begin{figure}[H]
\begin{center}
\includegraphics[scale=1]{input/problem_domain_analysis/Klassediagram_problem.pdf}
\caption{Statechart of a problem}
\label{fig:Klasse_diagram_problem}
\end{center}
\end{figure}

\subsection{Person}
As shown in figure \ref{fig:Klasse_diagram_person} a person is assigned a role in the system as soon as he is created. Thereafter multiple roles can be assigned to him.
\begin{figure}[H]
\begin{center}
\includegraphics[scale=1]{input/problem_domain_analysis/Klassediagram_person.pdf}
\caption{Statechart of a problem}
\label{fig:Klasse_diagram_person}
\end{center}
\end{figure}

\subsection{List}
Figure \ref{fig:Klasse_diagram_list} shows the statechart of the list class. The list is a container of problems, and it can only exist within a department. We chose not to include the list as a class, as it was nothing but a container.
\begin{figure}[H]
\begin{center}
\includegraphics[width=1\textwidth]{input/problem_domain_analysis/Klassediagram_list.jpg}
\caption{Statechart of a list}
\label{fig:Klasse_diagram_list}
\end{center}
\end{figure}

\subsection{Staff}
Figure \ref{fig:Klasse_diagram_staff} shows the statechart of the staff class. As shown on the Staff is a role to be assigned to persons. After a person recieves that role, he or she can start recieving problem assignments. When all problems are either unassigned or solved, the person will be idle, meaning that the person is not doing anything usefull with it's time. The role Staff can be unassgined only when there are no more problems assigned to that person.
\begin{figure}[H]
\begin{center}
\includegraphics[width=1\textwidth]{input/problem_domain_analysis/Klassediagram_staff.jpg}
\caption{Statechart of a problem}
\label{fig:Klasse_diagram_staff}
\end{center}
\end{figure}

\subsection{User}
The statechart for a user consists of working state in which the user can create problems and comment on them. This is shown by figure \ref{fig:Klasse_diagram_user}.
\begin{figure}[H]
\begin{center}
\includegraphics[width=1\textwidth]{input/problem_domain_analysis/Klassediagram_user.jpg}
\caption{Statechart of a problem}
\label{fig:Klasse_diagram_user}
\end{center}
\end{figure}

\subsection{Department}
As shown in figure \ref{fig:Klasse_diagram_department} can be open, after which staff can be both hired and fired. A department exists untill it is closed.
\begin{figure}[H]
\begin{center}
\includegraphics[scale=1]{input/problem_domain_analysis/klassediagram_department.pdf}
\caption{Statechart of a problem}
\label{fig:Klasse_diagram_department}
\end{center}
\end{figure}