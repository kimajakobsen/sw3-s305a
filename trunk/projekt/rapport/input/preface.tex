\chapter*{Preface}
\label{chap:preface}
\addcontentsline{toc}{chapter}{\numberline{}Preface}%
\emptyTop{}%
\vspace{-15mm}%
This report is written to document the 3\rd{} semester project by group s305a -- Software Engineering students from Department of Computer Science at Aalborg University. 
The project is commenced at \myDate{2}{9}{2010} and finished \myDate{17}{12}{2010}.
The overall theme of the project is programming. 
We have chosen to utilize the knowledge from the PE courses(project related courses) System Analysis and Design (SAD) and Object Oriented Programming (OOP), by designing our application using the Object Oriented Analysis \& Design method -- from here on called \ooad[] -- and programming \hdesk[] in the object oriented paradigm.  
Traditionally \ooad[] has a terms involving the ``system'', e.g. system component.
We are however making an application and are therefore exchanging these terms to an equivalent with the word ``application'' instead of ``system'', in order to keep consistency when talking about \hdesk[].

The report is split into five parts, \nameref{analysis}, \nameref{design}, \nameref{implementation}, \nameref{testing}, and \nameref{epilogue}.
Each part consist of several chapters.
In the top and bottom of every major chapter there is a small piece of text written in italic.
This marks the head and tail of each chapter.
The head outlines the content and the reason for the chapter to be in the report, whilst the tail summarizes the chapter.
In the end of the report there is an appendix which is recognized as the sixth part of the report.

Citations are written in square brackets i.e. [xx]. The number within the bracket is a reference to the bibliography which can be seen on page \pageref{chap:bib} in the appendix.
References to chapters or sections inside the report or in the appendix of this report are referred to either by the number of the given chapter or section, or by the page number where the given chapter or section is found.

The DVD contains the complete source code of \hdesk[], a complete database dump with sample data, the test project, and the report as a PDF file.  

When references to the code are made we use special notation for this which can be seen as follows: \vari{properties}, \vari{variables}, \cl{classes}, and \me{methods}.

We would like to give our thanks to Kristian Torp for help during setup of the database for our application and to Rene Hansen for helping us choose a framework to build our application upon.
Finally we would like to thank our supervisor Nadeem Iftikhar from Department of Computer Science at Aalborg University for continuous feedback and advice during creation of this report and our application.