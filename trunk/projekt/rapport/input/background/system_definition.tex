\section{System Definition}
\label{sec:systemdefinition}

A webbased system used to ease the collaboration between client, who has a problem, and technical staff, related to solving the problem, where the main emphasis is on getting the problems to the persons who are best at solving them, as efficient as possible. \\

%The system will automatic reassign problems if a staff member has become overloaded.

Before a problem is actually submitted to the system, the client is asked to \textit{categories} the problem by selecting tags from categories which define and describe the problem as specific as possible. Based on this categorization, the system will search for similar problems which solution might help the client, and therefore remove the need for submitting an identical or almost identical problem to the system.\\

If the client finds a problem which are equal or identical to his/hers own problem -- but without a solution -- the client is presented with the option to \textit{subscribe} to the problem, and receive notifications whenever new comments are made or a solution gets attached to the problem. \\

If the client should not find a solution to his/hers problem, he/she would be allowed to submit the problem, which would be assigned to an appropriate staff member, based on the clients selection of tags and categories, and the current workload of the staff members. \\

If a client can not determine the correct category for a specific problem, it will simply be submitted without the categorization, and the system will from there do one of the following three things, which a system administrator specifies:

\begin{itemize}
	\item distribute between all staff members, based on their workload
	\item distribute between all staff-personal associated with a specific department, based on their workload
	\item distribute to a single staff member, regardless of his workload
\end{itemize}

Every time a problem is assigned to any staff member, he or she will get a notification about it.

The system will generate priority sorted to-do lists for the staff. The priority will be based on a suggested priority by the client, and a final priority evaluation by the staff member assigned to the problem.
%The system will generate priority sorted to-do lists for the staff. The priority will be based on the category and the time since it were committed.

The system will make estimations based on time for solving problems of the same or somewhat equal categorization, the current workload of the assigned staff member, and the priority of the specific problem versus the other problems of which the specific staff member has to solve as well. 
% The estimation will be used to estimate a deadline. The staff member assigned to the problem will be notified when the deadline is near.\\

When problems are committed to the system and assigned to a staff member, the staff and client can add notes to the problem as a way of communication. Every time a change is made to a problem or a comment has been added, all associated clients and the assigned staff member gets notified.

\subsection{FACTOR criterion}
\label{sec:factor}


\paragraph{F (functionality)} Rankings, keep track, knowledge saved, Statistic time estimation, generate todo-lists.

\paragraph{A (Application domain)} People at the university

\paragraph{C (Conditions)} Only people associated with university, a webbased system, only people associated with the university, to-do-lists must be agile

\paragraph{T (Technology)} Web based, 

\paragraph{O (Objects)} Problems, solutions, staff, admin, client

\paragraph{R (Responsibility)} An administrative communication medium to ease communication between staff and client.\\