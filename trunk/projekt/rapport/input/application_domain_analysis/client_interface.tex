\subsection{Client Interface}
\label{sec:client_interface}

The client interface is illustrated in figure \ref{fig:client_interface}.
After login, the \aclient[] is presented with the \textit{main} screen, which gives three possibilities:
%After login, the \aclient[] is greeted with the \textit{main} screen, wich gives three posebilities:
\begin{itemize}
	\item Add problem
	\item My problems
	\item Search for problem(s)
	\item Statistics
\end{itemize}

These main possibilities are thoroughly described in the following sub-subsection, since they are the reflect the three use cases of the \aclient[] in section \ref{sec:usage}; submit problem, see all problems, and get statistics.
Due to the nature of our webbased system, the \aclient[] can anytime terminate the session and logout/close.

\subsubsection{Add Problem}
As the name indicates, this button initiates the process of adding a specific problem to the system.
%The process starts by selecting what kind of problem the \aclient[] has.

This is done by selecting \textit{tags} which describes the problem. Tags are grouped under categories. Each tag can only exist under one category, however if the need arises, it is possible to create a duplicate tag under another category. This does not mean that the same tag exists under two categories but two different tags with the same name exists under two different categories.\fixme{Er det smart at have to tags der hedder det samme?}
We say that the problem is ``categorized'' when it has tags associated with it.

When the \aclient[] is satisfied with the categorization of his/hers problem, he/she can click ``Search'', which will make the system search for problems with similar categorization, prioritizing those with a solution attached to it.
From there, the idea is that the \aclient[] might find a problem which is identical, or almost identical, which solution will also fix the \aclient 's problem. If such a problem is found, the \aclient[] has the option to \textit{``subcribe''} to the problem, making the \aclient[] receive all the same notifications as the person who created the problem.
This dramatically reduces redundancy in a case where multiple users has the same or similar problem.
This also saves the \aclient s the time to create and describe a new problem in the system.
If no suitable problem is found, the \aclient[] is allowed to describe his problem with words, as well as alter the tags which he/she selected earlier, in case he/she changed his/her mind.
Ultimately, the problem is added to the database after fully described. This process will initialize the workload monitor, for distributing the problem to a \astaff[] member, who is likely to solve the problem.

\subsubsection{My Problems}
This button navigates to a window where the problems created by the \aclient[] as well as the problems which the \aclient[] has subscribed to is shown.

\subsubsection{Search for Problem(s)}
Clicking this button will bring the \aclient[] to the problems search-screen, where he/she can search by specifying tags which possibly are attached to the problems that the \aclient[] wishes to find.

\subsubsection{Statistics}
\fixme{Der skal laves et interface til statistics, ellers det skal i hvert fald beskrives.}

\begin{figure}[htb]
\begin{center}
 \includegraphics[scale=0.70, clip=true, trim=0 6cm 0 0]{input/application_domain_analysis/client_interface}
\caption{\cinterface[]}
\label{fig:client_interface}
\end{center}
\end{figure}




