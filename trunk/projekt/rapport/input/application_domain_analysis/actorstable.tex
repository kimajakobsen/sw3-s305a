\begin{figure}[htdp]
\begin{center}
\begin{tabular}{l  ccccc}
\hline 
\multicolumn{2}{r}{\shf{Actor}} \\
\shf{Use case} &   \Aclient & \Astaff & \Wmon & \admin[c]  \\ \hline%
Submit problem 		  	& $\checkmark$ &  $\checkmark$  &  & $\checkmark$ \\ %
%Check problem status 	& $\checkmark$ & $\checkmark$ &  &\\ %
See all problems 		& $\checkmark$ & $\checkmark$  &  & $\checkmark$ \\ %
Balance workload 		&     &    &  $\checkmark$ & \\ %
Solve problem 			&     & $\checkmark$ &  & $\checkmark$ \\ %
%Add Tag				&     &  & & $\checkmark$ \\%
%Delete tag 				&     &  & & $\checkmark$ \\%
%Add category 			&     & & & $\checkmark$ \\%
%Delete category 		&     &  & & $\checkmark$ \\%
%Administrate Staff		&     &  & & $\checkmark$ \\ %
Administrate   &     &  & & $\checkmark$ \\%
Get Statistics   & $\checkmark$ & $\checkmark$ & & $\checkmark$ \\ \hline%

\end{tabular}
\end{center}
 \caption{\myCaption{Actor \& use case table}}

\label{tab:actoreventtable}
\end{figure}


Figure \ref{tab:actoreventtable} shows the relationship between use cases and the actors of our system. Note that \astaff{} members also are able to submit problems. This is due to the fact that both \aclient{} and \astaff{} simply are \textit{roles}. Persons associated with the role called \astaff{} can therefore also be associated with the role \aclient{}, which are able to submit problems to the systems.

The use cases in figure \ref{tab:actoreventtable} are described in subsection \ref{sec:usecase}.


