\section{Function}
\fixme{Vil vi have vores sections til at starte med sm\aa t eller stort?}
The purpose of this section is to ``determine the system's information processing capabilities'' \ref{roedeaalborg}. Ultimately resulting in ``a complete list of functions with specification of complex functions.'' 
\ref{roedeaalborg}
\fixme{ska ha ordnet bibtex}

Comprehensive knowledge of the OOA\&D method is assumed, and therefore we will not go in detail on function types etc. \fixme{overkill?}

\begin{table}[h] %tdp t = top , p = own page, d = unknown, b = bottom, h = here

\begin{center}
\begin{tabular}{|c|c|c|}
\hline
Add comment &   Simple & Read/signal   \\ \hline%
Reassign & Simple   & Update/signal \\ \hline%
Change status (problem) &   Simple & Update/signal \\ \hline%
Delete problem & Simple &   Update \\   \hline%
Search problems & Complex &   Read/Compute \\ \hline%
Create solution & Simple &   Update \\   \hline%
Attach solution & Simple &   Update/Signal \\   \hline%
Distribute problems &   Complex & Read/update/signal \\   \hline%
%Compare problems & Complex & Read/Compute \\ \hline%
Create problem &   Simple & Read/compute \\   \hline%
%Attach client to problem & Simple & Update \\ \hline%
Get staff todo list & Simple & Read \\   \hline%
Create Tag & Simple &   Update \\ \hline%
Create Category & Simple & Update \\ \hline%
Change visibility of Category &   Simple &   Update \\   \hline%
Change visibility of Tag &   Simple &  Update \\ \hline%
Attach tag to category & Simple & Update \\ \hline%
Detach tag to category & Simple & Update \\ \hline%
Create department & Simple & Update \\ \hline%
Delete department & Simple & Update \\ \hline%
Add staff & Simple & Update \\ \hline%
Remove staff & Simple & Update \\ \hline%
Assign staff to department & Simple & Update \\ \hline%
Get statistics & complex & Compute \\ \hline%
Subscribe client & simple & Update/signal \\ \hline%
Reattach tag & simple & Update \\ \hline%



\end{tabular}
\end{center}
\caption{Function list}
\label{tab:functionlist}
\end{table}

\emph{Add comment} simply adds a comment to a existing problem, making the comment visible for other \client and \astaff[] members. Marked as an \textit{signal}-function because a notification is sent to the assigned \astaff[] member. \fixme{Hvorfor er "add comment" markeret som "read"-funktion?} \\

\emph{Reassign} reassigns a specific problem from one assigned \astaff[] member to another. Marked as \textit{update}-fuction as it updates the model. Marked as a \textit{signal}-function as the new assigned \astaff[] member received a notification of the newly reassigned problem. \\

\emph{Change status (problem)} changes the status of a problem from one status to another. Again marked as a \textit{update}-function as it updates the model, also marked as a \textit{singal}-function as the associated \client s receives a notification. \\

\emph{Delete problem} simply deletes a problem. Marked as a \textit{update}-function as it updates the model. \\

\emph{Search problems} searches the model for problems matching a set of tags. Marked as ```medium'' complexity due to somewhat advanced algorithms.\fixme{Er det virkelig derfor den er medium?} Marked as a \textit{read}-function as it simply reads the model.\\

\emph{Create solution} simply creates solution to a problem and adds it to the model. Marked as a \textit{update}-function as it updates the model. \\

\emph{Attach solution} attaches a existing solution to a problem, and notifies the associated \client s about it. Marked as a \textit{update}-function as it updates the model. Marked as a \textit{signal}-function as it notifies the associated clients. \\

\emph{Distribute problems} is only ran by the workload monitor. It distributes newly created problems by assigning them to an appropriate \astaff[] member, based on current workload. % and specifications on what the different \astaff[] members are good at.
 Marked as a \textit{read}-function as it reads from the model. Marked as a \textit{update}-function as it updates the model. Marked as a \textit{signal}-function as it notifies the associated clients, and the newly assigned \astaff[] member. \\

%\emph{Compare Problems} compares a specific problem to the problems in the model and returns similair problems, prioritizing the ones with an attached solution, hence complex in complexity. Marked as a \textit{read}-function as it reads from the model, and marked as a \textit{compute}-function as a computation is involved involving information provided by the \client. \\

\emph{Create problem} simply creates a new problem and adds it to the model. Marked as a \textit{update}-function as it updates the model. \fixme{Hvorfor er "create problem" markeret som en compute-funktion?} \\

%\emph{Attach user to problem} simply attaches a new user to an existing problem. The need for this function arises when a \client has a problem wich is in need of a solution, but the problem has already been entered in the system by another \client. Marked as a \textit{update}-function as it updates the model. \\

\emph{Get staff todo list} simply fetches the \astaff[] members todo list and displays it. Hence marked as a \textit{read}-function. \\

\emph{Create tag} creates a tag, wich is an element used for categorizing problems, wich purpose is to make searching easier. Marked as a \textit{update}-function as it updates the model. \\

\emph{Create category} creates a category wich tags can be attached to. Marked as a \textit{update}-function as it updates the model. \\

\emph{Change visibility of category} changes the visibility of a category. Changes happen and a need of a category may vanish over time, but might be needed again in the future, thus making this function necessary. Marked as a \textit{update}-function as it updates the model. \\

\emph{Attach tag to category} simply attaches a specific tag to a specific category. Marked as a \textit{update}-function as it updates the model. \\

\emph{Detach tag from category} simply detaches a specific tag from a specific category. Marked as a \textit{update}-function as it updates the model. \\

\emph{Create department}. All \astaff[] members are associated with a department in order to ease the process of distributing newly created problems to appropriate \astaff[] members. Marked as a \textit{update}-function as it updates the model. \\

\emph{Delete department} as the name dictates, simply deletes a department. Marked as a \textit{update}-function as it updates the model. \\

\emph{Add staff} adds a specific \astaff[] member to the model. Marked as a \textit{update}-function as it updates the model. \\

\emph{Remove staff} removes a specific \astaff[] member from the model. Marked as a \textit{update}-function as it updates the model. \\

\emph{Assign staff to department} assigns a specific \astaff[] member to a specific department. Marked as a \textit{update}-function as it updates the model. \\

\emph{Get statistics} returns statistics, eg average time for specific kinds of problems to be solved, hence ``complex''-complexity. Marked as a \textit{compute}-function as it involves a fair amout of computation.
\emph{Subscribe client} subscribes a specific client to a specific problems. Marked as a \textit{update}-function as it updates the model. Marked as a \textit{signal}-function as the assigned \astaff[] member receives a notification. \\

\emph{Reattach tag} reattaches a specific tag from it's current category to another specified category. Marked as a \textit{update}-function as it updates the model.